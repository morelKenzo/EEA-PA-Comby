\documentclass[main.tex]{subfiles}
\begin{document}

On considère une variable aléatoire scalaire et réelle $Y$ de densité de probabilité :
\[ f_Y(y) = \frac{1}{X}e^{-\frac{y}{X}}u(x) \]
où $u(y)$ est la fonction échelon d'Heaviside et X un paramètre réel, inconnu, positif et supposé certain dans un premier temps.

\begin{enumerate}
\item Calcul de la valeur moyenne et de l'écart type de la VA $Y$
\begin{multicols}{2}
\begin{align*}
m_Y & = E[Y] = \int_{\mathbb{R}} y f_Y(y) dy \\
& = \int_0^{\infty} y \frac{1}{X} e^{-\frac{y}{x}} dy \\
& = [y(-e^{-\frac{y}{X}})]_0^{\infty} - \int_0^{\infty} (-e^{-\frac{y}{X}}) dy \\
& = \int_0^{\infty} (-e^{-\frac{y}{X}}) dy \\
& = [-Xe^{-\frac{y}{X}}]_0^{\infty} \\
& m_Y = X \\
\intertext{De plus,}
\sigma_Y^2 & = E[(Y-m_Y)^2] \\
& = E[Y^2] - m_Y^2 \\
E[Y^2] & = \int_0^{\infty} y^2\frac{1}{X}e^{-\frac{y}{X}}dy \\
& = [y^2(-e^{-\frac{y}{X}})]_0^{\infty} + 2X\int_0^{\infty}y\frac{1}{X}e^{-\frac{y}{X}}dy \\
& = 2Xm_Y = 2X^2\\
\sigma_Y^2 & = 2X^2 - X^2 = X^2 \\
\sigma_Y & = X
\end{align*}
\end{multicols}
\[ \boxed{m_Y = \sigma_Y = X} \]

\item On considère $N$ VA $Y_n, n=1..N$ indépendantes et identiquement distribuées.

On note $(y_1,...y_N)$ les réalisations de $(Y_1,...Y_N)$.\\

Grâce au résultat précédent $m_Y = X$, on peut estimer 
\[ \hat{x} = \frac{\sum_{n=1}^Ny_n}{N} \]
\[ \hat{X} = \frac{\sum_{n=1}^NY_n}{N} \]

De plus, l'estimateur est non biaisé car 
\[ E[\hat{X}] = \frac{\sum_{n=1}^N E[Y_n]}{X} = X \]

\item On souhaite exprimer la ddp conjointe des VA $Y_1,...Y_N$
\begin{align*}
f_\mathbf{Y}(\mathbf{y}) & = f_{Y_1, ... Y_N}(y_1,...y_N) \\
& = \prod_ {n=1}^N f_{Y_n}(y_n) \text{ par indépendance de } Y_1,...Y_N \\
& = \frac{1}{X^N} \exp(-\frac{\sum_ {n=1}^N Y_n}{X}) u(y_1)...u(y_N)
\end{align*}

\item On utilise l'estimateur du maximum de vraisemblance 
\[ \hat{x}_{MV} = \arg \max_X f_\mathbf{Y}(\mathbf{y}) \]
$\hat{x}_{MV}$ est la valeur de $x$ qui rend les valeurs $y_1,...y_N$ les plus probables.

Condition nécessaire (non suffisante) :
\begin{align*}
\frac{d f_{\mathbf{Y}} (\mathbf{y}) }{dX} |_{X = \hat{x}_{MV}} = 0 & \Leftrightarrow ...\\
& \Leftrightarrow (-\frac{N}{X} + \frac{\sum_{n=1}^N y_n}{X^2}) f_{\mathbf{Y}}(\mathbf{y}) |_{X = \hat{x}_{MV}} = 0 \\
& \Rightarrow \frac{-N}{\hat{x}_{MV}} + \frac{\sum_{n=1}^N y_n}{\hat{x}_{MV}^2} = 0 \\
& \Rightarrow \hat{x}_{MV} = \frac{\sum_{n=1}^N y_n}{N} \\
& \Rightarrow \hat{X}_{MV} = \frac{\sum_{n=1}^N Y_n}{N}
\intertext{Vérifier que c'est un maximum :}
\frac{d^2 f_{\mathbf{Y}} (\mathbf{y}) }{dX^2} > \text{ ou }& < 0 ? \\
\frac{d f_{\mathbf{Y}} (\mathbf{y}) }{dX} > 0 & \text{ pour } x\rightarrow 0^+ \\
\end{align*}

Calculons la moyenne et l'écart type de $\hat{X}_{MV}$
\begin{align*}
E[\hat{X}_{MV}] & = ... = X \\
\sigma_{MV}^2 & = E[(\hat{X}_{MV} - X)^2] \\
& = E[(\frac{\sum_{n=1}^N Y_n}{N} - \frac{NX}{N})^2] \\
& = E[(\frac{\sum_{n=1}^N (Y_n-X)}{N})^2] \\
& = \frac{1}{N^2} (\sum_{n=1}^N E[(Y_n-X)^2] + \sum_{i \neq j} E[(Y_i-X)(Y_j-X)] \\
& = \frac{NX^2}{N^2} \\
\sigma_{MV} & = \frac{X}{\sqrt{N}}
\end{align*}

\item On a beaucoup plus d'\textit{a priori} (d'informations sur X sans faire l'expérience) avec $\alpha = 1$ qu'avec $\alpha = 10$.

\begin{figure}[h!]
\centering
\begin{tikzpicture}[scale=2,samples=50,domain=0:5]
			\draw[-stealth] (0,0) -- (5,0) node[right] {$x$};
			\draw[-stealth] (0,0) -- (0,1.5) node[above] {$f_X(x)$};
			\draw plot (\x,{exp(-\x)});
			\draw[dashed] plot (\x,{0.1*exp(-\x*0.1)});
\end{tikzpicture}
\caption{Tracé de $f_X(x)$ pour $\alpha=1$ (--) et $\alpha=10$ (- -)}
\end{figure}

La courbe pour $\alpha=10$ est beaucoup plus étalée : \[ \sigma_{X,\alpha = 1} < \sigma_{X,\alpha = 10} \]



\item $\hat{x}_{MAP}  = \arg \max_X f_{X/\mathbf{Y} = \mathbf{y}} (x) $

Or,
\begin{align*}
f_{X/\mathbf{Y}   = \mathbf{y}} (x) & = f_{\mathbf{Y}, X = x} (\mathbf{y}) \frac{f_X(x)}{f_{\mathbf{Y}}(\mathbf{y})} \\
& = \prod_{n=1}^N f_{Y_i,X=x(y_i)}\frac{f_X(x)}{f_{\mathbf{Y}}(\mathbf{y})}  \\
& = g(x)u(y_1)...u(y_N)\frac{1}{f_{\mathbf{Y}}(\mathbf{y})}
\intertext{Ainsi,}
\hat{x}_{MAP} & = \arg \max_X f_{X/\mathbf{Y} =\mathbf{y}} (x) = \arg \max g(x)
\end{align*}

Condition nécessaire :
\begin{align*}
\frac{dg(x)}{dx}|_{x=\hat{X}_{MAP}} = 0 & \Leftrightarrow \hat{X}_{MAP} = \frac{-N\alpha + \sqrt{\alpha^2N^2 + 4\alpha \sum_{n=1}^Ny_n}}{2} \\
& \Leftrightarrow \hat{X}_{MAP} = \frac{-N\alpha + \alpha N \sqrt{1 + \frac{4}{\alpha N^2} \sum_{n=1}^Ny_n}}{2}
\end{align*}

\item Si $\alpha \rightarrow +\infty$, 
\begin{align*}
\hat{X}_{MAP} & \approx \frac{-N\alpha + \alpha N (1 + \frac{1}{2}\frac{4}{\alpha N^2} \sum_{n=1}^Ny_n}){2} \\
\hat{X}_{MAP} & \approx \frac{\sum_{n=1}^NY_n}{N} = \hat{X}_{MV}
\end{align*}
On n'a pas d'a priori sur X, on n'a que les observations.\\


Si $\alpha \rightarrow 0$, $m_X$ et $\sigma_X \rightarrow 0$ : $\hat{X}_{MAP} \rightarrow 0$.

L'a priori est fort.
\end{enumerate}
\end{document}
