\documentclass[main.tex]{subfiles}

\begin{document}
\newcommand{\snn}{\sum_n}
On cherche à estimer $V$ (paramètre constant). 

On relève la position du véhicule le long du rail à des instants $t_n=nT$.

On considère $Y_n=nTV+B_n$, avec $B_n~N(0,\sigma_B^2)$.

À $t_n=t_0=0$, le mobile se trouve en 0.

\section{Expériences}

\begin{enumerate}\setlength{\itemsep}{10mm}
\item On trace la droite qui passe au mieux par tous les points et l'origine, on trouve une pente d'environ 1 m/s.

L'hypothèse "bruit blanc" a l'air de marcher mais ne pas faire de conclusion rapide (on n'a que 10 mesures).

\item La "meilleure droite" ne passe pas par l'origine. Les causes possibles sont : un bruit non centré, ou une position non nulle à $t_0=0$. On a l'impression que le bruit est corrélé, mais on ne peut pas tirer de conclusion.

\item Pour obtenir la ddp de $\hat{V}$ :
\begin{itemize}
\item Méthode basée sur l'expérience : chaque jeu d'observation donne $\hat{v}_i$ et on trace l'histogramme (voir TP d'initiation à Matlab).
\item Méthode de changement de variable : ddp de $B_n$, puis ddp de $Y_n$ et enfin (passage difficile) ddp de $\hat{V}$.
\end{itemize}

Remarque : $\hat{V}~N(m_{\hat{v}},\sigma_{\hat{V}}^2)$
\begin{itemize}
\item 1er estimateur (non biaisé) : $m_{\hat{v}}=1m/s$ et $\sigma_{\hat{V}}=0,08m/s$
\item 2ème estimateur (non biaisé) : $m_{\hat{v}}=1m/s$ et $\sigma_{\hat{V}}=0,04m/s$, meilleur estimateur car meilleur écart-type.
\end{itemize}

\end{enumerate}

\subsection{Estimateur empirique}
Dans cette partie, $Y_n=nTV+B_n$ avec bruit faible, donc $V=\frac{y_n}{nT}$.

\begin{enumerate}\setlength{\itemsep}{10mm}
\item On mesure $y_n$. \[\hat{V}_{emp}=\frac{Y_n}{nT}\]
\[E[\hat{V}_{emp}] = \frac{E[Y_n]}{nT}=\frac{E[nTV+B_n]}{nT} = V+\frac{E[B_n]}{nT} = V\]
donc l'estimateur est non biaisé.
\[\sigma_{\hat{V}_{emp}}^2 = E[(\hat{V}_{emp}-m_{\hat{V}_{emp}})^2] = E[(\frac{Y_n}{nT}-V)^2] = E[(\frac{nTV+B_n}{nT}-V)^2] = \frac{1}{(nT)^2}E[B_n^2]\]
\[\sigma_{\hat{V}_{emp}}=\frac{\sigma_B}{nT}\]
Ainsi, pour minimiser $\sigma_{\hat{V}_{emp}}$, on prend $n=N$ (la plus grande mesure).

\item On dispose de $N$ mesures $y_1,...y_N$
\[\hat{V}_{emp}=\frac{\snn \frac{Y_n}{nT}}{n}\]
L'estimateur n'est pas biaisé car :
\[ E[\hat{V}_{emp}] = \frac{\snn \frac{E[Y_n]}{nT}}{N}=V\]
Écart-type de l'estimateur :
\begin{align*}
\sigma_{\hat{V}_{emp}}^2 & = E[(\hat{V}_{emp}-m_{\hat{V}_{emp}})^2]\\
& = E[(\hat{V}_{emp}-V)^2] \\
& = E[(\frac{\snn V+\frac{B_n}{nT}}{N}- \frac{NV}{N})^2] \\
& = \frac{1}{N^2}E[(\snn\frac{B_n}{nT})^2] \\
& = \frac{1}{(NT)^2}E[(\snn \frac{B_n}{n})^2] \\
& = \frac{1}{(NT)^2}(\snn\frac{E[B_n^2]}{n^2}+\sum_{n\neq m}\frac{E[B_nB_m]}{nm}) \\
\intertext{Le bruit est blanc, donc les $E[B_nB_m]=0$}
\sigma_{\hat{V}_{emp}}^2 & = \frac{\sigma_B^2}{(NT)^2}\snn\frac{1}{n^2} \\
\sigma_{\hat{V}_{emp}}  & = \frac{\sigma_B}{nT}\sqrt{S_1}
\end{align*}

Or, $\frac{\sigma_B}{nT}\sqrt{S_1} > \frac{\sigma_B}{nT}$. Cela signifie que notre estimateur avec 1 mesure est "meilleur" que celui avec $n$ mesures. On est triste d'avoir considéré les premières mesures qui sont très sensibles, comme nous, mais au bruit.
\end{enumerate}

\subsection{Préliminaires}
$V$ grandeur certaine mais inconnue, $Y_n=nTV+B_n$ et $B_n=N(0,\sigma_B^2)$.
\begin{enumerate}
\item Moyenne de $Y_n$ : $m_n = E[Y_n] = nTV$.

Écart-type de $Y_n$ : $\sigma_n^2=E[(Y_n-m_n)^2] = E[B_n^2]$ donc $\sigma_n=\sigma_B$

Coefficient de corrélation : \[\rho_{mn} = \frac{E[(Y_n-m_n)(Y_m-m_m)]}{\sigma_n\sigma_m} = \frac{E[B_nB_m]}{\sigma_B^2} = \left\{\begin{array}{cc} 1 & \si m=n \\ 0 & \si m\neq n \end{array}\right. = \delta_{n-m}\]

\item Par changement de variables aléatoires si ça t'amuse,
\[f_{Y_n}(y_n) = f_{B_n}(y_n-nTV) = \\frac{1}{\sqrt{2\pi \sigma_b^2}}e^{-frac{(y_n-nTV)^2}{2\sigma_b^2}}\]

Le caractère gaussien se conserve par transformation linéaire, i.e. toute combinaison linéaire de VA suivant une loi gaussienne suit aussi une loi gaussienne. Attention, ne pas sommer les ddp.

Les $Y_i$ étant indépendants (car les $B_i$ sont indépendants car blancs) :
\[f_{Y_1,...Y_n}(y_1,...y_n)=\prod_{i=1}^N f_{Y_i}(y_i)=\frac{1}{(2\pi)^{N/2}\sigma_B^N}\exp(-\frac{1}{2}\frac{\snn(y_n-nTV)^2}{\sigma_B^2})\]
\[f_{\mathbf{Y}}(\mathbf{y}) =\frac{1}{(2\pi)^{N/2}\sigma_B^N}\exp(-\frac{1}{2}\frac{(\mathbf{y}-E[\mathbf{y}])^T(\mathbf{y}-E[\mathbf{y}])}{\sigma_B^2})\]
\end{enumerate}

\noindent Rappel : la décorrélation n'implique pas l'indépendance, il faut le caractère "gaussiens dans leur ensemble".

\subsection{Estimateur des moindres carrés}

\begin{enumerate}\setlength{\itemsep}{10mm}
\item L'estimateur au sens des moindres carrés de $V$ est :
\[ \hat{v}_{MC} = \arg_V \min \sum_{n=1}^N (y_n-nTV)^2\] 

En posant $J_{MC} : V \rightarrow \sum_{n=1}^N(y_n-nTV)^2$, $J_{MC}$ est une parabole (concavité tournée vers le haut). On a alors une condition nécessaire et suffisante à la minimisation :
\begin{align*}
\frac{dJ_{MC}}{dV}|_{V =\hat{V}_{MC}} = 0 & \Leftrightarrow \sum_{n=1}^N 2(-nT)(y_n-nTV) = 0 \\
& \Leftrightarrow \sum_{n=1}^N ny_n - \hat{v}_{MC} T \sum_{n=1}^N n^2 = 0 \\
& \Leftrightarrow \hat{v}_{MC} = \frac{\snn ny_n}{T\sum_{n=1}^N n^2} \\
& \Leftrightarrow \hat{V}_{MC} = \frac{\snn nY_n}{TS_2}
\end{align*}

\item Calculons la moyenne de l'estimateur.

\[m_{MC} = E[\hat{v}_{MC}] = \frac{\snn n E[Y_n]}{T S_2} = \frac{\snn n (nTV)}{T S_2} = V\]
L'estimateur non biaisé.\\

On s'intéresse à son écart-type.
\begin{align*}
\sigma_{MC}^2 & = E[(\hat{V}_{MC}-m_{MC})^2]\\
& = E[(\hat{V}_{MC}-V)^2] \\
& = E[(\frac{\snn nY_n}{TS_2}-V)^2] \\
& = E[(\frac{\snn n^2 (TV + nB_n) - TV\snn n^2}{TS_2})^2] \\
& = E[(\frac{\snn nB_n}{TS_2})^2] 
\intertext{Comme les $B_n$ sont décorrélés, les doubles produits sont tous nuls}
\sigma_{MC}^2 & = \frac{\snn n^2 E[B_n^2]}{T^2 S_2^2} \\
\sigma_{MC} & = \frac{\sigma_B}{T\sqrt{S_2}}
\end{align*}

Comme $S_2=\snn n^2 > N^2$, on a $\sqrt{S_2} > N$ donc $\sigma_{MC} < \sigma_{emp} = \frac{\sigma_B}{NT}$.

Notre estimateur est meilleur que l'estimateur empirique.
\end{enumerate}

\newpage
\subsection{Estimateur du maximum de vraisemblance}

\begin{enumerate}\setlength{\itemsep}{10mm}
\item L'estimateur du maximum de vraisemblance de $V$ est donné par 
\[ \hat{v}_{MV} = \arg_V \max f_{\textbf{Y}}(\mathbf{y}) \]

Remarque : dans le cas où $V$ est incertain, $\arg_v \max f_{\mathbf{Y}/V=v}(\mathbf{y})$.\\

$Y$ suit une loi gaussienne donc :
\[ \hat{v}_{MV} = \arg_V \max f_{\textbf{Y}}(\mathbf{y}) = \arg_V \min \snn (y_n-nTV)^2 = \arg \min J_{MC}(V)\]

\item Identique à la partie précédente car $\hat{V}_{MV} = \hat{V}_{MC}$.
\end{enumerate}

\subsection{Estimateur du maximum a posteriori}
\begin{itemize}
\item $V$ suit une loi gaussienne $N(V_0,\sigma_V^2)$
\item les VA $B_n$ et $V$ sont indépendantes 2 à 2
\end{itemize}
\medskip
On a 2 types d'informations :
\begin{itemize}
\item celle qui vient des mesures $y_n$
\item celle qui vient de l'a priori $V$
\end{itemize}

\begin{enumerate}\setlength{\itemsep}{10mm}
\item Dans le cas où $V=v$ ($v$ est certain), on ne change pas pour autant le comportement de $B_1,...B_n$, donc de $Y_1...Y_n$ :
\[ f_{\mathbf{Y}/V=v}(\mathbf{y}) = \frac{1}{(2\pi)^{N/2}\sigma_B^N}\exp(-\frac{1}{2}\frac{\snn(y_n-nTv)^2}{\sigma_B^2}) \]

\item On utilise la règle de Bayes :
\[f_{V/\mathbf{Y}=\mathbf{y}}(v)  = \frac{f_{\mathbf{Y}/V=v}(\mathbf{y}) f_V(v)}{f_{\mathbf{Y}}(\mathbf{y})} \]
$f_{\mathbf{Y}}(\mathbf{y})$ ne dépend pas de $v$ car on peut la calculer selon $f_{\mathbf{Y}}(\mathbf{y}) = \int_{\R} f_{\mathbf{Y},V}(\mathbf{y},v)dv$.

\item L'estimateur du maximum a posteriori de $V$ est donné par :
\[\hat{v	}_{MAP} = \arg_v \max f_{V/\mathbf{Y}=\mathbf{y}}(v)\]

Or \[ f_{V/\mathbf{Y}=\mathbf{y}}(v) = cste \times \exp(-\frac{1}{2}(\frac{\snn (y_n - nvT)^2}{\sigma_B^2} + \frac{(v-V_0)^2}{\sigma_V^2})) \]

On pose $J_{MAP} = \frac{\snn (y_n - nvT)^2}{\sigma_B^2} + \frac{(v-V_0)^2}{\sigma_V^2}$ et on a alors $\hat{v}_{MAP} = \arg_v \min J_{MAP}(v)$.

CNS de maximisation : 
\begin{align*}
\frac{dJ_{MAp}}{dV}|_{V =\hat{v}_{MAP}} = 0  & \Leftrightarrow -2T \frac{\snn (y_n - nvT)n}{\sigma_B^2} + 2\frac{(v-V_0)}{\sigma_V^2} = 0 \\
& \Leftrightarrow \hat{v}_{MAP} = \frac{\frac{T\snn ny_n}{\sigma_B^2}+\frac{V_0}{\sigma_V^2}}{\frac{T^2 \snn n^2}{\sigma_B^2} + \frac{1}{\sigma_V^2}} \\
& \Leftrightarrow \hat{v}_{MAP} = \frac{\frac{\hat{v}_{MV}}{\sigma_{MV}^2} + \frac{V_0}{\sigma_V^2}}{\frac{1}{\sigma_{MV}^2} + \frac{1}{\sigma_V^2}}
\end{align*}

Donc 
\[ \hat{V}_{MAP} = \frac{\frac{\hat{V}_{MV}}{\sigma_{MV}^2} + \frac{V_0}{\sigma_V^2}}{\frac{1}{\sigma_{MV}^2} + \frac{1}{\sigma_V^2}} \]
C'est un barycentre entre les mesures représentées par $\hat{v}_{MV}$ et l'a priori $V_0$.

\begin{itemize}
\item Si $\sigma_V$ "petit", alors $\hat{V}_{MAP} \approx V_0$ : beaucoup d'a priori donc on n'a pas exploité les mesures.
\item Si $\sigma_V$ "grand", alors $\hat{V}_{MAP} \approx \hat{V}_{MV}$ : l'a priori est tellement pourri qu'on n'en tient pas compte.
\end{itemize}

\item On forme l'erreur d'estimation 
\[ \tilde{V}_{MAP} = \hat{V}_{MAP} - V \]

\begin{itemize}\setlength{\itemsep}{10mm}
\item Biais ? 
\[ E[\tilde{V}_{MAP}] = E[\hat{V}_{MAP}] - E[V] = 0 \]

\item Variance de l'erreur d'estimation : puissance de l'erreur dans le cas non biaisé.
\begin{align*}
\sigma_{MAP}^2 & = E[ (\tilde{V}_{MAP} - E[\hat{V}_{MAP}])^2] \\
& = E [(\hat{V}_{MAP} - V)^2] \\
& = E [ (\frac{\frac{\hat{V}_{MV}-V}{\sigma_{MV}^2} + \frac{V_0-V}{\sigma_V^2}}{\frac{1}{\sigma_{MV}^2} + \frac{1}{\sigma_V^2}})^2]\\
& = \frac{1}{(\frac{1}{\sigma_{MV}^2} + \frac{1}{\sigma_V^2})^2} E [(\frac{\hat{V}_{MV}-V}{\sigma_{MV}^2} + \frac{V_0-V}{\sigma_V^2})^2] \\
& = \frac{1}{(\frac{1}{\sigma_{MV}^2} + \frac{1}{\sigma_V^2})^2}( \frac{E[(\hat{V}_{MV}-V)^2]}{\sigma_{MV}^4} + \frac{E[(V_0-V)^2}{\sigma_V^4}) \\
& = \frac{1}{(\frac{1}{\sigma_{MV}^2} + \frac{1}{\sigma_V^2})^2}( \frac{\sigma_{MV}^2}{\sigma_{MV}^4} + \frac{\sigma_V^2}{\sigma_V^4}) \\
\sigma_{MAP}^2 & = (\frac{1}{\sigma_{MV}^2} + \frac{1}{\sigma_V^2})^{-1}
\end{align*}
On a donc $ \sigma_{emp} > \sigma_{MV} > \sigma_{MAP}$.
\end{itemize}

\end{enumerate}


\end{document}
