\documentclass[main.tex]{subfiles}
\begin{document}
\subsection*{Optimisation d'un laminoir}

\begin{enumerate}
\item Y désigne la longueur de barres perdue, qui est de X (la totalité de la barre) si la barre est trop courte, ou de X-l si la barre est trop longue.
\[ 
Y = \left\{ 
\begin{array}{ll}
X & \text{ si } X < l \\
X-l & \text{ si } X >l 
\end{array}
\right.
\]

\item 
\begin{align*}
m_Y & = E_Y[Y] = E_X[Y] \\
& = \int_0^lxf_X(x)dx + \int_l^{\infty} (x-l)f_X(x)dx \\
& = \int_0^{\infty} xf_X(x)dx - \int_l^{\infty}lf_X(x)dx \\
m_Y & = m_X - l\int_l^{\infty}f_X(x)dx
\end{align*}

\item On suppose que X est une variable aléatoire gaussienne, alors \[f_X(x) = \frac{1}{\sigma_X\sqrt{2\pi}}e^{-\frac{(x-m_X)^2}{2\sigma_X^2}} \]
\begin{itemize}
\item $m_X$ représente la moyenne de $X$, c'est-à-dire la valeur sur laquelle la gaussienne est centrée
\item $\sigma_X$ représente la dispersion des valeurs autour de $m_X$. En effet, à $m_X \pm \sigma_X$, on a $f_X(x) = 0.6\frac{1}{\sigma_X\sqrt{2\pi}}$.
\begin{align*}
\int_{m_X-\sigma_X}^{m_X+\sigma_X} f_X(x)dx = 0.68 \\
\int_{m_X-2\sigma_X}^{m_X+2\sigma_X} f_X(x)dx = 0.95 \\
\int_{m_X-3\sigma_X}^{m_X+3\sigma_X} f_X(x)dx = 0.99
\end{align*}
\end{itemize}
Le modèle est valable si la probabilité d'avoir des événements "non-physiques" reste faible, c'est-à-dire si $m_X-3\sigma_X > 0$ de sorte à ce que $P(X<0) < 0.5 \%$.

\item Sans contrainte sur l'intervalle de variation de $m_X$, une condition nécessaire à l'obtention d'un minimum $m_X^{opt}$ est $\frac{dm_Y}{dm_X} = 0$.
\begin{align*}
\frac{dm_Y}{dm_X} &  = 1 - l\int_l^{+\infty} \frac{\partial f_X(x,m_X)}{\partial m_X}dx \\
&\quad \text{Or, on a ici une gaussienne, donc } \frac{\partial f_X}{\partial m_X} = - \frac{\partial f_X}{\partial x} \\
\frac{dm_Y}{dm_X} & = 1 + l\int_l^{\infty} \frac{df_X(x,m_X)}{dx}dx \\
& = 1 + l(f_X(+\infty)-f_X(l)) = 1 - lf_X(l) \\
\text{Ainsi, } \frac{dm_Y}{dm_X} & = 0 \quad \Leftrightarrow \quad  f_X(l) = \frac{1}{l} 
\end{align*}

\begin{align*}
f_X(l) = \frac{1}{l} & \Leftrightarrow 1 = \frac{l}{\sqrt{2\pi}\sigma_X}e^{-\frac{1}{2}(\frac{x-m_X}{\sigma_X})^2} \\
& \Leftrightarrow m_X = l \pm \sigma_X\sqrt{-2\ln\frac{l}{\sqrt{2\pi}\sigma_X}} \text{ avec } l > \sqrt{2\pi}\sigma_X
\end{align*}

Pour qu'il s'agisse d'un minimum, il faut que $\frac{d^2m_Y}{d^m_X} > 0$.
\begin{align*}
\frac{d^2m_Y}{d^m_X} & = -l \frac{df_X(l)}{dm_X} \\
& = \frac{-l}{\sqrt{2\pi}\sigma_X^3}(l-m_X)e^{-\frac{1}{2}(\frac{x-m_X}{\sigma_X})^2} \\
\text{ et on obtient } \frac{d^2m_Y}{d^m_X}|_{m_X+} & > 0 \text{ et } \frac{d^2m_Y}{d^m_X}|_{m_X-} & < 0
\end{align*}

Le minimum de $m_Y$ est donc bien atteint pour $m_{Xopt} = l + \sigma_X\sqrt{-2\ln\frac{l}{\sqrt{2\pi}\sigma_X}}$

\item Application numérique : \\
\[ m_X = 1 + 0,01\sqrt{2\ln\frac{1}{\sqrt{2\pi}0,01}} = 1,027 m \]
Remarque : on obtient $l \approx l + 3\sigma_X$, donc le modèle est valable, en accord avec la question 4.
\end{enumerate}

\subsection*{Changement de variable aléatoire}
\begin{enumerate}
\item Traçons l'évolution de la VA $Y$ en fonction de $X$ : \\
\begin{tikzpicture}

\draw [>=latex,<->] (8,0) node[right]{$x$} -- (0,0) -- (0,5) node[left]{$y$};

\draw [dashed] (0,0) -- (2,2);
\draw [dashed] (3,3) -- (5,5) node[right]{$y=x$};
\draw (2,2)-- (3,3);

\draw[dashed] (4,1) -- (6,3) node[right]{$y=x-l$};
\draw (3,0) -- (4,1);

\draw[dashed] (0,2) node[left]{$l-\Delta$} -- (2,2) -- (2,0) node[below]{$l-\Delta$};
\draw[dashed] (0,3) node[left]{$l$} -- (3,3) -- (3,0) node[below]{$l$};
\draw[dashed] (4,1) -- (4,0) node[below]{$l+\Delta$};

\draw[dashed,blue] (-0.5,2.6) node[left]{$y\in[l-\Delta,l]$} -- (2.6,2.6) -- (2.6,-0.5) node[below]{$y$};
\draw[dashed,red] (-0.5,0.3) node[left]{$y\in[0,\Delta]$} -- (3.3,0.3) -- (3.3,-0.5) node[below]{$y+l$};
\end{tikzpicture}

Avec l'aide du dessin, on voit immédiatement que :
\begin{itemize}
\item si $y\in[0,\Delta[$, alors $F_Y(y) = P[0 \leq Y < y] = P[l \leq x < y+l] = F_X(y+l) - F_X(l)$
\item si $y\in[l-\Delta,\Delta]$, alors $F_Y(y) = P[0 \leq Y < y] = F_X(l+\Delta) - F_X(l) + F_X(y) - F_X(l-\Delta)$
\item sinon, $F_Y(y) = $ cste.
\end{itemize}

Ainsi, on en déduit la densité de probabilité en dérivant par rapport à $y$ :
\[
f_Y(y) = \left\{
\begin{array}{ll}
f_X(y) & \text{ si } y\in[l-\Delta,\Delta] \\
f_X(y+l) & \text{ si } y\in[0,\Delta[ \\
0 & \text{ sinon}
\end{array}
\right.
\]


\item On a le changement de variable $g: X \Rightarrow Y=g(X)$ où :
\[
g(X) = \left\{
\begin{array}{ll}
X+l & \text{ si } Y = X-l \in [0,\Delta] \\
X & \text{ si }Y = X \in [l-\Delta,\Delta]
\end{array}
\right.
\]
On utilise la formule générale du changement de variables :
\[ f_Y(y) = F_X(x) |\frac{dx}{dy}| |_{x,g(x)=y} \]

On retrouve ainsi le résultat précédent.

\item On considère que la variable $X$ est uniformément répartie sur l'intervalle $[l-\Delta,l+\Delta]$. Ainsi, $f_X(x) = C$ (constante) sur cet intervalle et comme on a :
\[ \int_{-\infty}^{+\infty} f_X(x) dx = \int_{l-\Delta}^{l+\Delta} f_X(x) dx = 2\Delta C = 1 \]
alors 
\[ f_X(x) = \left\{
\begin{array}{ll}
\frac{1}{2\Delta} & \text{ si } x \in [l-\Delta,l+\Delta] \\
0 & \text{ sinon }
\end{array}
\right.
\]
d'où
\[ f_Y(y) = \left\{
\begin{array}{ll}
\frac{1}{2\Delta} & \text{ si } y \in [0,\Delta]\cup[l-\Delta,l] \\
0 & \text{ sinon }
\end{array}
\right.
\]

\textbf{Propriété importante :} Si une densité de probabilité a une symétrie en $x_0$ et que sa moyenne existe, sa moyenne est sur l'axe de symétrie, donc vaut $x_0$.\\
Preuve : 
\begin{align*}
\int_{-\infty}^{+\infty} x f_X(x) dx & = \int_{-\infty}^{+\infty} (y+x_0) f_X(y+x_0) dy \\
& = x_0 \int_{-\infty}^{+\infty} f_X(y+x_0)dy + \int_{-\infty}^{+\infty} y f_X(y+x_0)dy \\
& = x_0 \times 1 + 0 = x_0
\end{align*}

Ainsi, on obtient $m_Y = \frac{l}{2}$

\item On considère le changement de variable : $\Phi \rightarrow Y = \cos(\Phi)$.

Comme $\cos(\Phi) \in [-1,1]$, on a pour $y < -1, F_Y(y) = 0$ et pour $y \geq 1, F_Y(y) = 1$.

Pour $y\in[-1,1]$, 
\begin{align*}
F_Y(y) & = P(\cos(\Phi) < y) \\
& = P(-pi \leq \Phi < -\arccos(y)) + P(\arccos(y) \leq \Phi < \pi)\\
& = F_{\Phi}(-\arccos(y)) - F_{\Phi}(\arccos(y)) + F_{\Phi}(\pi) - F_{\Phi}(-\pi) \\
& = F_{\Phi}(-\arccos(y)) - F_{\Phi}(\arccos(y)) + 1\\
f_Y(y) & = \frac{1}{\sqrt{1-y^2}}f_{\Phi}(-\arccos(y)) + \frac{1}{\sqrt{1-y^2}}f_{\Phi}(\arccos(y))
\end{align*} 

\item On utilise le résultat général sur les changements de VA avec $\phi = \arccos(y)$ si $y>0$ et $\phi = -\arccos(y)$ si $y<0$
\begin{align*}
f_Y(y) & = f_{\Phi}(\phi) |\frac{d\phi}{dy}||_{\phi,g(\phi)=y} \\
& = \left\{
\begin{array}{ll}
0 & \text{ si } y \notin [-1,1] \\
f_{\Phi}(\phi) |\frac{d\phi}{dy}||_{\phi=\arccos(y)} + f_{\Phi}(\phi) |\frac{d\phi}{dy}||_{\phi=-\arccos(y)}& \text{ si } y\in[-1,1]
\end{array}
\right. \\
& = \left\{
\begin{array}{ll}
0 & \text{ si } y \notin [-1,1] \\
\frac{1}{\sqrt{1-y^2}}f_{\Phi}(-\arccos(y)) + \frac{1}{\sqrt{1-y^2}}f_{\Phi}(\arccos(y))&  \text{ si } y\in[-1,1]
\end{array}
\right.
\end{align*} 

\item Si  la VA$\Phi$ est uniformément répartie sur $[-\pi,\pi]$, alors on a 
\begin{align*}
f_{\Phi}(\phi) & = 
\left\{
\begin{array}{ll}
\frac{1}{2\pi} & \text{si } \phi \in [-\pi,\pi] \\
0 & \text{ sinon }
\end{array}
\right. \\
f_Y(y) & = 
\left\{
\begin{array}{ll}
\frac{1}{\pi\sqrt{1-y^2}} & \text{si } y \in [-1,1] \\
0 & \text{ sinon }
\end{array}
\right. \\
\end{align*}
\end{enumerate}

\end{document}
