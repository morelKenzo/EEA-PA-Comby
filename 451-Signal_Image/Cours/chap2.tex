\documentclass[main.tex]{subfiles}
\begin{document}
\section{Stationnarité et ergodicité}
\begin{defin}
  Soit $(\Omega,\mathcal{E},P)$ un espace probabilisé.
  Une famille/suite de VA indexé par le temps est un \emph{signal aléatoire} $\in\C^n$ noté : $X_t(\omega) ~ \forall t\in \R$ (ou $X_n(\omega) ~ \forall n \in \Z$)
\end{defin}

En ptratique on s'interesse àdes des signaux de dimension 1.
\paragraph{Rappel:} on appelle trajectoire la réalisation / acquisition d'un signal. il existe deux types de moyenne possible:
\begin{itemize}
\item temporelle, idem que celle des signaux déterministes
\item Statistique, ideme que pour les VA.
\end{itemize}
\begin{exemple}Soit le SA suivant:
  $X(t,\omega) =A \sin(2\pi f_0 t)$ où $A$ est une variable aléatoire , (ici qui suit une loi uniforme).
  Alors une réalisation de ce SA est $x(t)=a\sin(2 \pi f_0 t)$.
  \begin{itemize}
  \item $\overline{x(t,\omega)} = 0 = m_x$ et $\overline{x^22(t,\omega)} = \frac{a^2}{2}$
  \item $E[X(t,\omega)] =\sin(2 \pi f_0 t)E[A] = m_X(t)$.
  \end{itemize}
\end{exemple}
\subsection{Moyenne temporelle}
On rappelle les différentes expression des 1er et 2nd ordre (si il existe) de trajectoire particulière.
\begin{defin}
  Les moments d'ordre 1 temporel sont des \emph{moyennes temporelle}:
  \begin{itemize}
  \item Temps continu
    \[
      \overline{x(t,\omega)}=\lim_{T\to\infty}\frac{1}{2T}\int_{-T}^{T}x(t,\omega)\d t = m_x(\omega)
    \]
  \item Temps discret
    \[
      \overline{x[n,\omega]} =\lim_{N\to\infty}\frac{1}{2N+1} \sum_{n=-N}^{N}x[n,\omega] =m_x(\omega)
    \]
  \end{itemize}
\end{defin}


\begin{defin}
  Les moments d'ordre 2 croisés définissent la fonction d'intercorrélation temporelle (($\omega$ est fixé )
  \begin{itemize}
  \item Temps continu:
    \[
      \overline{x(t,\omega)\cdot y^*(t-\tau,\omega)} = \lim_{T\to\infty}\frac{1}{2T}\int_{-T}^{T}x(t,\omega) y^*(t-\tau,\omega)\d t =C^p_{xy}(\tau,\omega)
    \]
  \item Temps discret:
    \[
\overline{x[n,\omega]y^*[n-k,\omega]} =\lim_{N\to\infty}\frac{1}{2N+1} \sum_{n=-N}^{N}x[n,\omega]y^*[n-k,\omega]
    \]
  \end{itemize}
\end{defin}

\begin{rem}
On dit également que les  Les moments temporels dépendent de la trajectoire.

Si $y=x$ on parle d'autocorrélation. De plus $C_{xx}^p(0)$ est la \emph{puissance de $x$}.
\end{rem}
\subsection{Ergodicité}
\begin{defin}
  \begin{itemize}
  \item Un processus est \emph{ergodique au sens stricte}
    si et seulement si toutes les moyennes temporelles sont indépendantes de la trajecoire considérée.
  \item
    Un processus est ergodique à l'ordre $n$ si et seulement si tous les moments jusqu'à l'ordre de $n$ sont indépendant de la trajectoire considéré.
    Les moments temporel d'un signal ergodique ne sont pas des variables aléatoires.
\end{itemize}
\end{defin}
\begin{rem}
  Souvent $n=2$ Pour 2 SA on parle d'ergodicité dans leur ensemble.
\end{rem}

\subsection{Moyenne statistique}

On considère les signaux aléatoire à des instants particuliers, fixé.

\begin{rem}
  En fixant le temps on peux définir les fonctions de répartition et la densité de probabilité  d'un signal aléatoire. Alors on peux exprimer les moments statistiques de ses signaux temporels:
\end{rem}
\begin{defin}
  On défini la moyenne statistique (moment d'ordre 1):
  \[
    m_X(t) = E[X(t,\omega)] = \int_{\R}^{}x f_X(x,t) \d x
  \]
  et la fonction d'intercorrélation statistique (moment d'ordre 2):
  \[
    \gamma_{xy}(t_1,t_2) =E[X(t_1,\omega)y^{*}(t_2,\omega)] = \iint xy^{*}f_{x,y,t_1,t_2}\d x\d y
  \]
  Il en est de meme dans le cas discret.
\end{defin}
\subsection{Stationnarité}
\begin{defin}
  \begin{itemize}
  \item Un processus aléatoire est\emph{ stationanaire au sens strict} ssi
    toutes ses caractéristiques statistiques sont invariantes par tout
    changement de l'origine des temps.
  \[
    f_{X}(x,t) = f_X(x,t+\tau) =f_X(x) \quad \forall \tau 
  \]
\item Un processus aléatoire est stationnaire au sens large /au second ordre ssi ses moments d'ordre 1 et 2 sont invariants par tout changement d'origine des temps.
  \[
    E[|X(t,\omega)|^2] = E[|X(t',\omega)|^2] < +\infty
  \]
\end{itemize}
\end{defin}
\subsection{Stationnarité et ergodicité}
\begin{prop}
  Si  un SA est à la fois stationnaire et ergodique les moyennes temporelles et statistiques sont égales.

  L'ensemple des processus stochastique,stationnaire, ergodique peux être obtenu à partir d'une seule trajectoire allant de $-\infty$ à $+\infty$.
\end{prop}
\begin{prop}
  Un SASE au second ordre est tel que:
  \[
    m_x = E[X(t)]=\overline{x(t)}=m_x
  \]
  et
  \[
    \gamma_{xx}(\tau) = E[X(t)X^{*}(t-\tau)]=\overline{x(t)x^{*}(t-\tau)} = C_{xx}^p(\tau)
  \]
\end{prop}
\section{Corrélation et densité spectrale de puissance}
Ici on s'interesse la répartition de la puissance d'un SA en fonction de la fréquence (idem que la DSE pour des signaux  à énergie finie). On se restreint à des SAS du 2nd ordre.
\paragraph{Notation} :
$x(t,\omega)$ représente le SA ou une des ses réalisation \\
$X(f)$ représente la TF d'un signal $x$ sous réserve d'existence.
\begin{thm}[Wiener-Kintchine]
  \[
    TF[\gamma_{xx}]=\Gamma_{XX}(f) = \text{ dsp de x(t)}
  \]
\end{thm}

\begin{prop}[Cas du TC]
  \begin{align*}
    \Gamma_{xy}(f) &= \int_\R \gamma_{xy}(\tau) e^{-j2\pi ft} \d t\\
    \gamma_{xy}(\tau) & =\int_\R \Gamma_{xy}(f) e^{j 2\pi f\tau} \d f\\
   \gamma_{xy}(0) & = \int_\R \Gamma_{xy}(f) \d f  = \text{ puissance (statistique)}\\
  \end{align*}
\end{prop}
\begin{prop}[Cas du TD]
  \begin{align*}
    \Gamma_{xy}(f) &=\sum \gamma_{xy}[k] e^{-j 2 \pi fk}\\
    \gamma_{xy}[\tau] & = \int_{-1/2}^{1/2} \Gamma_{xy}(f)e^{j 2\pi f k} \d f\\
    P_x = \gamma_{xx}[0] &= \int_{-1}^{1} \Gamma_{xx}(f)df
  \end{align*}
\end{prop}
\begin{exemple} cf TP1
\end{exemple}

\begin{prop}
  \begin{itemize}
  \item $|\gamma_{xy}(\tau)| \le \gamma_{xx}(0) =P_x$ 
  \item $\gamma_{xx}(-\tau)= \gamma_{xx}(\tau)^* \implies \Gamma_{xx}(f) \in \R$
  \item $\Gamma_{xx}(f)>0$
  \end{itemize}
\end{prop}
\begin{rem}
  très souvent on a $\gamma_{xx}(\tau) \xrightarrow[+\infty] |m_x|^2$ , ce qui signifie qu'on a aps d'effet  ``mémoire'' àl'infini. Si $m_x \neq 0$ la DSP  comporte une raie à l'origine de valeur $m_x$.
\end{rem}

\section{Periodogramme}



\section{Signaux aléatoire particulier}
\subsection{SA indépendants}
\subsection{SA décorrélés}

\end{document}

%%% Local Variables:
%%% mode: latex
%%% TeX-master: "main"
%%% End:
