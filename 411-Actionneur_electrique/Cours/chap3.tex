\documentclass[main.tex]{subfiles}
\begin{document}
\section{Intro}
\section{Modèle de commande}

\section{Couplage réseau}

\[
  \deriv[(i)]{t} = -[L]^{-1}[R](i) - [L]^{-1}\left\{(v)-p\Omega \deriv[(\Phi_0)]{t}\right\}
\]
On a donc égalité des amplitudes et des phases pour la vitesse:
\[
  \boxed{(v) = p\Omega \deriv[(\Phi_0)]{t}}
\]


\section{Schéma équivalent Behn-Eschenburg}
\paragraph{Hypothèse}:
\begin{itemize}
\item RPS sinus
\item MS non saturé , pole lisse , éuilibré
\end{itemize}
\begin{center}

\begin{tabular}[c]{rl}
  \begin{minipage}[c]{0.3\linewidth}
\begin{circuitikz}
  \draw (0,0) to[V] ++(0,2) to[L] ++(2,0) to[R] ++(2,0) to[open,v<=$V$] ++(0,-2) -- (0,0);
\end{circuitikz}
\end{minipage}

&
\begin{minipage}[h]{0.5\linewidth}

\[
  \underline{E} = (j\mathcal{L}\omega+R) \underline{I} + \underline{V}
\]
\end{minipage}

\end{tabular}
\end{center}
\subsection{Diagramme de Fresnel}
On considère l'origine de phase sur $V$ et on se place en alternateur ie $\delta = Arg(E)-Arg(V) > 0$

\subsubsection{Surexcitation}

[schema fresnel]
\begin{prop}
$
\|E\| > \|V\|
$ on a alors :
\[
  \begin{cases}
    P > 0 \\
     Q >0 \\
  \end{cases}
\]
\end{prop}
\subsubsection{Sousexcitation}
[schema fresnel]
\begin{prop}
$
\|E\| < \|V\|
$ on a alors :
\[
  \begin{cases}
    P > 0 \\
     Q >0 \\
  \end{cases}
\]
\end{prop}

\end{document}
