\documentclass[main.tex]{subfiles}
\begin{document}
\section{Principe fondamentaux: de la cellule de commutation au bras d'onduleur}
\subsection{Principes}

\begin{itemize}
\item Conversion statique (Énergie électrique $\to$ Énergie électrique):
  adapter les tensions, les courants ( mettre  en forme, modifier les amplitudes) pour gérer les transferts de puissances.
\item Connexion séquentielle en commutation

\begin{center}
	\begin{tabular}{|c|p{5cm}|p{5cm}|}
		\hline
		\diagbox{Entrée}{Sortie} & DC & AC \\
		\hline
		AC &  Redresseur (non) commandés &  Gradateurs Cyclo-convertisseurs\\
		\hline
		DC &  Hacheurs alimentation à découpage & Onduleurs de tension commutateur de courant\\
		\hline
	\end{tabular}
\end{center}
\item Composants
  \begin{itemize}[label = $-$]
  \item Sources d'alimentation électrique (tension et courant)
  \item Élements passifs (Inductance, transformateur, condensateur , PAS de résistances)
  \item Interrupteur de puissance
  \end{itemize}
\end{itemize}

\subsection{Sources d'alimentation électrique}
Il existe théoriquement 2 type de sources:
\begin{itemize}
	\item source de tension
	\item source de courant
\end{itemize}
pour deux régimes de fonctionnement
\begin{itemize}
	\item régime statique
	\item régime dynamique/ instantanée.
\end{itemize}

\subsubsection{Régime statique}
\paragraph{source de tension}
\begin{defin}
  Une source de tension impose la tension quelque soit le courant
  et on a
  \[\lim\limits_{f\to0} \left|\frac{\delta V}{V_0}\right| << \lim\limits_{f\to0}\left|\frac{\delta I}{I_0}\right|\]
\vspace{-2em}
  \begin{center}
    \begin{circuitikz}
      \draw (0,0) to[V, v=$V_v$, i=$i_v$] (2,0);
    \end{circuitikz}
  \end{center}
\end{defin}
\paragraph{Source de courant} ~
\begin{defin}
 Une source de courant impose le courant quelque soit la tension à ses bornes à puissance limitée	et on a
  \[\lim\limits_{f\to0} \left|\frac{\delta I}{I_0}\right| << \lim\limits_{f\to0}\left|\frac{\delta V}{V_0}\right|\]
\vspace{-2em}
  \begin{center}
    \begin{circuitikz}
      \draw (0,0) to[V, v=$V_v$, i=$i_v$] (2,0);
    \end{circuitikz}
  \end{center}

\end{defin}
\paragraph{Source instantanées}
\begin{description}
\item[de tension] ~
  \begin{defin}
   une source instantanée de tension est un  dipôle capable de limiter les variations de tension en présence de variation instantanée de courant.
    \[\lim\limits_{f\to \infty}\left|\derivp[V_v]{I_v}\right|_{V_0,I_0}  << \left|\frac{V_0}{I_0}\right|\]
\vspace{-2em}
    \begin{center}
      \begin{circuitikz}
        \draw (0,0) to[R,i=$i_v$, v=$V_v$] (2,0);
      \end{circuitikz}
    \end{center}
  \end{defin}
\item[De courant] ~
  \begin{defin}
   une source instantanée de courant tension est un  dipôle capable de limiter les variations de tension en présence de variation instantanée de courant.
   \[\lim\limits_{f\to \infty}\left|\derivp[V_v]{I_v}\right|_{V_0,I_0}  << \left|\frac{V_0}{I_0}\right|\]
   \vspace{-2em}
    \begin{center}
    \begin{circuitikz}
      \draw (0,0) to[R,i=$i_v$, v=$V_v$] (2,0);
    \end{circuitikz}
  \end{center}
  \end{defin}
\end{description}

\paragraph{Remarque}
Toutes les sources "réelles" sont limitées en puissance.
\subsubsection{Règle d'association}
	\paragraph{Pour une source de tension}
	\begin{itemize}
		\item jamais en court-circuit
		\item peut être ouverte
	\end{itemize}
	\paragraph{Pour une source de courant}
	\begin{itemize}
		\item jamais ouverte
		\item peux être court-circuitée
	\end{itemize}
\paragraph{Exemple de sources Statique selon leur réversibilité}

\begin{center}
\begin{tabular}{|c|>{\centering\arraybackslash}p{3cm}|>{\centering\arraybackslash}p{3cm}|}
			\hline
			& réversible en tension & irréversible en tension \\
			\hline
			réversible en courant & machine électrique & batterie \\
			\hline
			irréversible en courant &  & pile \\
			\hline
\end{tabular}
\end{center}
\subsection{Interrupteur de puissance}

On utilise des semi-conducteur de puissance pour construire des interrupteurs de puissances.
\begin{center}
  \begin{circuitikz}
    \draw    (0,0) to[spst,i=$i_k$] ++(2,0);
    \draw (0,-0.5) to [open,v<=$v_k$] ++(2,0);
  \end{circuitikz} \\
$K$ fermé : $v_k= 0$, $i_k\neq0$, \\ $K$ ouvert $v_k\neq0$, $i_k=0$
\end{center}

\begin{prop}
C'est la commutation qui dissipe de la puissance :
\[
  w_k = \int_{t_{com}}^{}v_k(t)i_k(t) \ge 0
\]
\end{prop}
\subsubsection*{Exemple d'interrupteur de puissance}

diode , transistor IGBT, mosfet

à chaque fois , caractéristique statique, symbole , convention fléchage

Le transistor IGBT fonctionnent aux alentour de 10kHz

\subsection{Règle d'association des sources}
\begin{defin}
  un interrupteur:
\begin{itemize}
	\item ne doit jamais court-circuiter une source de tension
	\item peux ouvrir une source de tension
	\item ne doit jamais ouvrir une source de courant
	\item peux court-circuiter une source de courant
\end{itemize}

\end{defin}


\paragraph{Exemple}
\begin{figure}[H]
	\centering
	\begin{circuitikz}
		\draw (0,0) to[V,v=$V$] (0,2) to[switch,l=$K_1$] (2,2)  to [switch] (2,0);
		\draw (0,0) -- (4,0) to[I,l=$K_2$ i<=$i$] (4,2)-- (2,2);
	\end{circuitikz}
	\caption{Cellule de Commutation }
\end{figure}
Les deux interrupteurs fonctionnent en opposition pour respecter les règles d'associations.

\emph{C'est la structure de base d'association de source ! }

\section{Conversion DC- AC}
\subsection{Introduction}
Les onduleurs de tension sont très variés ( large plage de fréquence, frequence, et/ou tension variable ...)
\subsubsection{Modulation de largeur d'impulsion}

on controle la structure suivante:
\begin{figure}[H]
  \centering
  \begin{circuitikz}
    \draw (0,1) node[nigbt,bodydiode](A){$k_1$}
    (0,-1) node[nigbt,bodydiode](B){$k_2$};
    \draw (A.E) -- (B.C)
    (A.C) |- ++(-3,0.5)
    (B.E) |- ++(-3,-0.5)++(0,-0.2) to[open, v^=$U_{DC}$] ++(0,5)
    (0,0) to[short,i^=$i_s$,-o] ++(1,0) to[open, v^<=$v_s$] ++(0,-2)
    ;
    \draw (A.B) ++(-2,0) to[amp] (A.B) (B.B)++(-2,0) to[amp,mirror] (B.B);
  \end{circuitikz}
  \caption{ Cellule de commutation commandée}
\end{figure}
\begin{defin}
  On définit une fonction de modulation tel que :
\[f_m(t)=
  \begin{cases}
    1 & \implies v_s =U_{DC}\\
    0 & \implies v_s = 0  \\
  \end{cases}
\]
\end{defin}
\begin{prop}
  On a en sortie

  \[
    \begin{cases}
      i_s= f_m I_{DC}
      v_s = f_mU_{DC}
    \end{cases}
  \]
\end{prop}
\begin{description}
\item[MLI naturelles]

Hysterisis
\item[MLI calculée, répétée]

Lecture de table, MLI vectorielle, comparaison avec triangle.
\end{description}

\subsubsection{Grandeur filtrée et moyennée}
On rappelle la définition d'une valeur moyenne:
\begin{defin}
  \[
    X = <x(t)> = \frac{1}{T_{dec}}\int_{T_dec}^{}x(t)dt
  \]

\end{defin}

\begin{prop}[Cas de la MLI]
  On a le rapport cyclique
  \[
    \alpha = \frac{m(t)}{A}
  \]
  alors :
\[V_S = <v_s(t)> = U_{DC}<f_m(t)> = \alpha U_{DC} =\frac{m(t)}{A}U_{DC}\]
\end{prop}

\subsection{Structure d'onduleur monophasé}
\paragraph{objectif :} Piloter $v_s(t)$ ,avec les contraintes suivantes:
\begin{itemize}
\item $\alpha\in[0,1]$
\item $A =1$
\item $m(t) = \frac{1}{2}+\frac{m_0}{2}sin(\omega_0t)$
\end{itemize}
On a alors :
\[
  \boxed{V_s(t) = \frac{U_{DC}}{2}}+\frac{U_{DC}}{2} m_0sin(\omega_0t)
\]

\subsubsection{Montage en demi-pont}


\begin{figure}[H]
  \centering
  \begin{circuitikz}
    \draw (0,0) |- ++(1,1.5) to[amp] ++(2,0) coordinate(A1){}  ++(0.6,0) node[nigbt,bodydiode](A){}
    (0,0) |- ++(1,-1.5) to[amp,-o] ++(2,0) coordinate(A2){}  ++(0.6,0) node[nigbt,bodydiode](B){};
    \draw (A1)--(A.B) (A2)--(B.B) (A.E) -- (B.C) coordinate[midway](M);
    \draw (A.C) -- ++(2,0) to[V,v<=$\frac{U_{DC}}{2}$] ++(0,-2)
    (B.E) -- ++(2,0) to[V,v_=$\frac{U_{DC}}{2}$] ++(0,2) -- ++(0,0.6);
    \draw (M) to[I,v^=$v_0$] ++(2,0) ;
  \end{circuitikz}
  \caption{Structure en demi-pont}
\end{figure}

La tension est sinusoidale pure dans la charge :
\[
\boxed{v_o(t) = (2f_m-1)\frac{U_{DC}}{2} = \pm \frac{U_{DC}}{2}}
\]

\begin{enumerate}[label=\alph*)]
\item pleine onde :
  \begin{center}
    \begin{tikzpicture}
      \begin{axis}[axis lines=middle
        ,samples=41,
        domain = 0:1.5,
xmin=0,ymin=-2,xmax=1.5,ymax = 2,
ticks=none,
]
\addplot+[no marks] {1.2*sin(2*pi*deg(x)};
\addplot+[no marks] plot coordinates {(0,1) (0.5,1) (0.5,-1) (1,-1) (1,1) (1.5,1)};
      \end{axis}
    \end{tikzpicture}
  \end{center}

  \begin{prop}
    On a
    $V_{oeff}   =\frac{U_{DC}}{2}$ et $V_{oeff}' = \frac{4}{\pi}\frac{U_{DC}}{2\sqrt{2}} \simeq 48\% U_{DC}$ \\
    On a un THD de 48\%.
  \end{prop}


\item MLI :
  \begin{align*}
    V_0(t) &= V_0sin(\omega t) \text{ et } f_0 \ll f_{dec} \\
    m(t) &= \frac{A}{2}+\frac{V_0}{U_{DC}}sin(\omega_0t)\\
    \alpha(t) &= \frac{1}{2} + \frac{V_0}{U_{DC}}sin(\omega_0t)
  \end{align*}

  On définit :
  \begin{defin}
    \begin{description}
    \item[N] Indice de modulation $\frac{f_{dec}}{f_0} > 1$
    \item[r] taux de modulation $\frac{2V_0}{U_{DC}} <1 $
    \end{description}
  \end{defin}
  l'analyse spectrale de $v_0(t)$ donne:
\end{enumerate}

  \begin{figure}[H]
    \centering
    \begin{tikzpicture}
      \begin{axis}[axis lines=middle,width=15cm,height=7cm,
        domain = 0:1.5,
xmin=0,ymin=0,xmax=12,ymax = 1.5,
ytick=\empty,
xtick={1,9,10,11},
xticklabels={$f_0$ , $f_d-f_0$ ,$f_d$ , $f_0+f_d$},
]
\draw[-latex](axis cs:1,0) -- (axis cs:1,1);
\draw[-latex](axis cs:10,0) -- (axis cs:10,1);
\draw[-latex](axis cs:9,0) -- (axis cs:9,0.8);
\draw[-latex](axis cs:11,0) -- (axis cs:11,0.8);
      \end{axis}
  \end{tikzpicture}

  \caption{On a tout interet à prendre $N>>1$}
\end{figure}
\subsubsection{Montage en pont complet}
cette fois ci on a le montage:
\begin{prop}
  $v_{s1} = f_{m1}U_{DC} $ et $v_{s2}= f_{m2} U_{DC} $
  \[
    v_0= (f_{m1}-f_{m2})U_{DC}
  \]
\end{prop}
\begin{enumerate}[label=\alph*)]
\item Commande bipolaire

  \begin{defin}
    Pour une commande bipolaire on a besoin que d'une fonction de modulation:
    \[
      f_{m2} = 1-f_{m1} = \overline{f_{m1}}
    \]
  \end{defin}
\item Commande unipolaire
  \begin{itemize}
  \item pleine onde

    \begin{prop}
      Avec une commande bipolaire sur un pont complet on a:
      \begin{itemize}
      \item amplitude $2\times$ plus grande qu'en 1/2 pont.
       \item courant non sinus
       \item pas de réglage d'amplitude
      \end{itemize}
    \end{prop}

  \item MLI
  \end{itemize}

\item Commande unipolaire (3 états)
  \begin{defin}
    En commande unipolaire, $f_{m1} \neq f_{m2}$ et on peux avoir trois états pour la charge.
  \end{defin}
\end{enumerate}


\section{Onduleur de tension triphasé}


\subsection{Structure}
[Schéma]

\subsection{Commande}
\begin{itemize}
\item pleine onde
  \emph{cf TD3}
\item MLI
\end{itemize}

\subsection{Vue de la charge triphasé équilibrée, neutre non relié}
\begin{center}

\begin{tabular}{ll}
  \begin{minipage}[h]{0.3\linewidth}

\begin{circuitikz}
  \draw (0,0) to[R] ++(2,0);
  \draw (0,1) to[R] ++(2,0)node[right]{N'};
  \draw (0,2) to[R] ++(2,0);
  \draw (2,0) -- (2,2);
\end{circuitikz}
\end{minipage}
&
                  \begin{minipage}{0.5\linewidth}
On a les équations :
\[
  \vect{v_{1N'} \\  v_{2N'} \\v_{2N'}} = \frac{U_{DC}}{3}
  \begin{bmatrix}
    2& -1 &-1 \\
    -1 &2 &-1 \\
    -1& -1&2
  \end{bmatrix}
  \vect{f_{m1} \\f_{m2}\\f_{m3}}
\]
\end{minipage}

\end{tabular}
\end{center}

et :
\[
m_i = \frac{A}{2}+\frac{Ar}{2}\sin\left(\omega_0t-(i-1)\frac{2\pi}{3}\right)
\]
puis:
\[
  v_{iN'} = r
  \frac{U_{DC}}{2}
  \sin\left(\omega_0t-(i-1)\frac{2\pi}{3}\right)
\]

Alors :
\begin{align*}
  V_{0fonda}^{eff} &= \frac{1}{\sqrt{2}}\frac{2}{2\pi}\int_{0}^{2\pi}V_0(\theta+\beta/2)\cos(\theta)d\theta\\
                   &=\frac{4U_{DC}}{\sqrt{2}\pi}\int_{0}^{\beta/2}\cos(\theta)d\theta\\
                   &=\frac{4U_{DC}}{\sqrt{2}\pi} \sin(\beta/2)
\end{align*}
\paragraph{MLI}:
  1 porteuse, 2 modulantes

\end{document}
