\documentclass[main.tex]{subfiles}
\begin{document}
\begin{rem}
  Le role del'égaliseur n'est pas le même en transmission numérique et analogique.
\end{rem}
En numérique on utilise un égaliseur pour garantir le respect du critère de nyquist.
\section{Egaliseur numérique }

On rappele le schéma de chaine de transimission numérique:
\begin{figure}[H]
  \centering
  \begin{tikzpicture}
    [every node/.style={draw,rectangle,minimum height=4em,node distance=0.5cm,scale=0.8}]
    \node (CS){\begin{tabular}{c}Codage \\ source\end{tabular}};
    \node (CC)   [right= of CS]{\begin{tabular}{c}Codage \\ canal\end{tabular}};
    \node (CBB)  [right= of CC]{\begin{tabular}{c}Emission \\ $G(f)$\end{tabular}};
    \node (C)    [right= of CBB]{
      \begin{tabular}{c}
Canal\\ H(f)
      \end{tabular}
};
    \node (A)    [right= of C][adder]{};
    \node (Demod)[right= of A]{
      \begin{tabular}{c}Reception\\ Gr(f)
      \end{tabular}
};
    \node (E)    [right= of Demod]{
      \begin{tabular}{c}
Egaliseur \\(E)
      \end{tabular}
};
    \node (Decod)[right= of E]{Detecteur};
    \tikzset{every node/.style={}}
    \draw (S) -- (CS) -- (CC) -- (CBB)-- (C) -- (A.1) (A.3) -- (Demod) -- (E) -- (Decod);
    \draw[latex-] (A.4) -- ++(0,1) node[above]{Bruit};
  \end{tikzpicture}
  \caption{Principe d'une chaine de transmission numérique}
\end{figure}
\begin{prop}

On considère que le canal de transmission est idéal:
\[
  h(t) = K. \delta(t-\tau) \text{ soit } H(f) = K.e^{-2j\pi f\tau}
\]
Alors :
\begin{itemize}
\item Signal en sortie du canal n'est pas déformé
\item Si l'impulsion du canal vérifie le critère de Nyquist, en se
  placant au meme rythme d'échantillonnage T pour ensuite detecter les
  différents niveau correspondant à un code m-aire.
\end{itemize}
\end{prop}
\begin{rem}
  Pour un canal quelconque on a le bruit , l'attenuation, une bande limitée... Tout cela peux conduire à une erreur de décodage sur les echantillons.

  On place donc un \emph{egaliseur} pour compenser ces effets dans la chaine de reception du signal.
\end{rem}
\section{Réglage de l'égaliseur}
\begin{rem}
  Le rôle de l'égaliseur n'est pas le meme suivant le type de transmission (analogique/numérique).
  \begin{itemize}
  \item En transmission analogique on veux :
    \[
      H(f).E(f) = exp(-2\pi f\tau)
    \]
    Pour compenser le retard dans le canal de transmission pour qu'il soit idéal du point de vue du recepteur.
  \item Pour une transmission numérique :

    Il faut que l'impulsion perçu respecte (après l'égaliseur)le premier critère de Nyquist.
  \end{itemize}
\end{rem}

\begin{prop}
Pour respecter le critère de Nyquist en numérique il faut que:
  \[
\sum_{n} G\left(f-\frac{n}{T}\right) \cdot H\left(f-\frac{n}{T}\right) \cdot G_{r}\left(f-\frac{n}{T}\right) \cdot E\left(f-\frac{n}{T}\right)=T
\]
\end{prop}

\begin{rem}
  POur une impulsion issue d'un filtre rectangulaire
  \[
G\left(f-\frac{n}{T}\right) . H\left(f-\frac{n}{T}\right) \cdot G_{r}\left(f-\frac{n}{T}\right) \cdot E\left(f-\frac{n}{T}\right)=T \cdot \operatorname{rect}_{1 / T}(f)
\]
\end{rem}

\begin{rem}
  \begin{itemize}
  \item Comme nous l’avons vu au chapitre précédent, on choisit plutôt
    un impulsion de Nyquist.
  \item   L’égaliseur est implémenté numériquement
    et s’apparente à un filtre numérique.
  \item   Différentes stratégies
    d’optimisations sont possibles (Moindres carrés, adaptatifs,
    etc...).
\end{itemize}
\end{rem}

ajout sur le diagramme de l'oeil....
\end{document}

%%% Local Variables:
%%% mode: latex
%%% TeX-master: "main"
%%% End:
