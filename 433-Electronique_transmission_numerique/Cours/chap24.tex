\documentclass[main.tex]{subfiles}
\begin{document}
Dans cette partie on étudie l'influence du canal sur le signal.
\begin{defin}
  On rappel:
  \begin{itemize}
  \item La rapidité de modulation est noté $R = 1/T$ en [bauds]\\
  \item Le débit binaire $D = 1/T_b$.
  \item $T = log_2(M) .T_b$
  \end{itemize}
  Soit:
  \[
    R = \frac{D}{\log_2(M)}
  \]
\end{defin}

\section{Caractéristique du canal}
\label{sec:carac_canal}
On choisit d'étudier un canal :
\begin{itemize}
\item linéarie et invariant (caractérisé par sa réponse impulsionnelle  $g(t)$, sa réponse fréquentielle $G(f)$ ...)
\item bruité par un bruit $n(t)$ additif.
\item de type passe-bas et de bande $B$.
 \item associé à un filtre de réception de réponse impulsionnelle $g_r(t)$.
\end{itemize}

Le signal recu et filtré par le fitre de réception:
\begin{align*}
  r(t) &= g_r(t) \ast h(t) \ast e(t) + g_r(t)\ast n(t)\\
       &= g_r(t) \ast h(t) \ast \sum_{k}^{}a_kg(t-kT)+ b(t)\\
       &= \sum_{k}^{}a_ky(t-kT)+b(t)
\end{align*}
$b(t)$ représente la contribution totale du bruit après filtrage.

\begin{prop}
  On considère que le bruit est additif blan gaussien (BABG) àmoyenne nulle et de variance $\sigma^2$
  \[_B(b) = \frac{1}{\sqrt{2\pi\sigma^2}}e^{\frac{-b^2}{2\sigma^2}}\]
\end{prop}
\begin{prop}
  Le filtre de reception peux être optimiser afin de maximiser le rapport signal sur bruit après réception:
  \[
    G_r^{opt}=(G(f).H(f))^*
  \]
\end{prop}
\begin{proof}
  
\end{proof}
Ainsi après échantillonnage à l'instant de décision on a :
\[
  r(t_0+nT) = \sum_{k}^{}a_ky(t_0+nT-kT)+b(t_0+nT)= d(t)
\]
soit:
\[
  r(t_0+nT) = a_ny(t_0)+\sum_{k\neq n}^{}a_ky(t_0+(n-k)T)+b(t_0+nT)
\]
\begin{defin}
  On défini le terme d'interférence entre symbole comme:
  \[
IES =  \sum_{k\neq n}^{} a_k y(t_0+(n-k)T)
\]
Que l'on peux exprimer comme:
\[
  \sum_{k\neq n}^{} a_k y(t_0+(n-k)T) = \sum_{k}^{}a_kg_r(t_0+nT)\ast h(t_0+nT)\ast g(t_0+(n-k)T)
\]
\end{defin}

\begin{prop}
  En considérant un récepteur parfaitement synchronisé on souhaite qu'à l'instant de prise de décision :

\[
  r(t_0+nT)  = a_n y(t_0)+ b(t_0+nT)
\]
Soit $IES = 0 $
\end{prop}

\begin{rem}
  Dans le cas d'un filtre de réception optimal, et pour une synchronisation parfaite, l'annulation de l'IES consiste à choisir une forme d'impulsion compatible avec le canal et telle que l'IES soit nulle.
\end{rem}
\section{Premier critère  du Nyquist}

\begin{thm}[Critère de Nyquist]
  On ne peux pas transmettre sans IES un signal de rapidité de modulation $R = 1/T$ dans une bande inférieure à $1/2T$.
  Un canal respectant le premier criètre de Nyquist est tel que:
  \[
    B \ge \frac{1}{2T} = B_{Nyquist}
  \]
\end{thm}
\begin{defin}
  On appelle Bande de \textsc{Nyquist} la bande minimale pour une durée de symbole $T$.
  \[
    B_{Nyquist} = \frac{1}{2T}
  \]
\end{defin}

\begin{proof}
  On considère un canal décrit en \ref{sec:carac_canal}. Sans bruit on a :

  \[
    r(t_0+nT) = a_ny(t_0)+\sum_{k\neq n}^{}a_k y(t_0+(n-k)T)
  \]
  On souhaite obtenir:
  \[
    r(t_0+nT) = a_ny(t_0) \implies y(t_0+nT) =
    \begin{cases}
      y(t_0) & text{ pour } n = 0\\
      0 & \forall n \neq 0 \\
    \end{cases}
  \]

  En sortie de l'échantillonneur on a la prise de décision :
  \[
    d(t) = y(t)\sum_{n}^{}\delta(t-t_0-nT)
  \]
  Soit dans l'espace de Fourrier:
  \[
    D(f) = \frac{1}{T}Y(f-\frac{n}{T})e^{-j2\pi n t_0/T} \tag{(*)}
  \]
  De plus on a également:
   \[
     d(t) = \sum_{n}^{}y(t_0+nT)\delta(t-t_0-nT)
   \]
   Soit dans l'espace de Fourrier:
   \[
     D(f) = \sum_{n}^{}y(t_0+nT)e^{-j 2 \pi f(t_0+nT)} \tag{(**)}
   \]
   Par unicité de la transformée de Fourier on a :
   \[
     \sum_{n}^{}Y(f-\frac{n}{T}) e^{-j2\pi (f-n/T)t_0} = T y(t_0)
   \]
   alors:
   \[
     Y^{(t_0)}(f)  = \frac{Y(f)}{y(t_0)}e^{j 2 \pi f t_0}
   \]

   Le premier critère de Nyquist s'écrit donc:
   \[
     \sum_{n}^{}Y^{(t_0)}(f-\frac{n}{T}) =T 
   \]
 \end{proof}

\section{Impulsion de Nyquist}

  Toutes les fonctions qui vérifie l'équation suivante, vérifie le critère de Nyquist.
  \[
    \sum_{n}^{}Y^{(t_0)}(f-\frac{n}{T}) = T
  \]

  La DSP rectangulaire centrée en fait partie.
  \begin{rem}
    On a cepedant des lobs secondaire élevé dans la DSP (sinus
    cardinal pur), ce qui peux être dramatique en cas de mauvaise
    synchronisation. On cherche donc d'autre fonctions candidates avec
    des lobes secondaires moins élevé.
  \end{rem}

  \begin{defin}
    Le filtre en cosinus surélevé vérifie ces deux critères. Sa DSP
    est alors:
    \[G(f) = 
      \begin{cases}
        T  & \forall f  \in \left[-\frac{1-\alpha}{2T},\frac{1-\alpha}{2T}\right]\\
        \frac{T}{2}\left[1+\sin\left(\frac{\pi T}{\alpha}\left(\frac{1}{2T}-|f|\right)\right)\right] & \frac{1-\alpha}{2T}\le |f| \le \frac{1+\alpha}{2T}\\
        0 & \text{ sinon}
      \end{cases}
    \]
    Ce qui donne la reponse temporelle:
    \[
      g(t) = \frac{\sin\left(\frac{\pi t}{T}\right)}{\frac{\pi t}{T}}
      \frac{\cos\left(\frac{\pi\alpha t}{T}\right)}{\left(1-4\alpha^2 \frac{t^2}{T^2}\right)}
    \]

    Avec$\alpha$ le \emph{Roll-off} compris entre 0 et 1 .

    Pour $\alpha=0$on retrouve l'impulsion rectangulaire. pour $\alpha=1$ on a
    un filtre de Hanning.
  \end{defin}
  \begin{exemple}
    Dans la téléphonie 3G $\alpha=0,22$.
  \end{exemple}

  \begin{figure}[H]
    \centering
    \begin{subfigure}{0.5\textwidth}
          
      \caption{Réponse fréquentielle}
     
    \end{subfigure}%
    \begin{subfigure}{0.5\textwidth}
      \centering
      \caption{réponse temporelle}
     
    \end{subfigure}
    \caption{Différents filtres respectant le critère de Nyquist}
  \end{figure}
 
  \section{Capacité de canal}

  \subsection{Critère empirique HTS}
  \begin{defin}
    \textsc{Hartley}, \textsc{Tuller},\textsc{Shannon} on établi la
    formule empirique:
    \[
      m \le m_{max} = \sqrt{1+\frac{S}{N}}
    \]
    Avec $S$ et $N$ puissance du signal et du bruit. et $m$ le nombre
    de niveau de codage possible.
  \end{defin}
  \subsection{À retenir}
  \begin{defin}
    C la capacité du canal est le nombre maximal de bit qu'il est
    susceptible de transmettre par seconde.

    \[C = D_{max} = R.\log_2(M) = B\log_2\left(1+\frac{S}{N}\right)
    \]
  \end{defin}

  \begin{prop}
    Pour un canal de transmission de type passe-bas, de bande passante
    $B$ et bruité par un BABG le débit doit toujours être inférieur à
    la capacité de Shannon.
  \end{prop}
\end{document}
