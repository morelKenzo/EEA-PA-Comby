\documentclass[main.tex]{subfiles}
\newcommand{\snzi}{\sum_{n=0}^{+\infty}}
\newcommand{\snii}{\sum_{n=0}^{+\infty}}
\begin{document}

\section{Cellule de base}

La structure d'une cellule de base donnée ci-dessous, fait appel à deux \emph{switches}, chacun commandé par une horloge.

\begin{figure}[H]
  \centering
  \begin{circuitikz}
    \draw (0,0)node[left]{$x_c(t)$} to[spst,l=horloge1] (2,0) to[spst,l=horloge2] (4,0)
    (2,0) to[C,v<=$x_e(t)$] (2,-2) node[ground]{}
    ;
  \end{circuitikz}
  \caption{Cellule de commutation}
  \label{fig:commut}
\end{figure}

\begin{figure}[H]
  \centering
  \begin{tikzpicture}
     \begin{axis}%
       [axis lines = middle,
       at = {(0,0)},
      height = 5cm,width =8cm,
      xlabel = {$t$},
      ylabel = {$|S(f)|$},
      ytick=\empty,
      xmin = 0,xmax = 10, ymin = -0.1, ymax = 1.5,
      xtick = {3,6,8.5},
      xticklabels = {$\frac{T_e}{2}$,$T_e$,$T_e+\tau_1$},
      ]
      \addplot[no marks, black] plot coordinates {(0,1)(2.5,1)(2.5,0)(6,0)(6,1)(8.5,1)(8.5,0)};
      \addplot[no marks, black, dotted] plot coordinates
      {(3.5,0) (3.5,1) (5.5,1)(5.5,0)};
      \node at (axis cs:1.25,0.5){$h1$} ;
      \node at (axis cs:4.5,0.5){$h2$} ;
    \end{axis}
  \end{tikzpicture}
\caption{Signaux horloges utilisés}
\end{figure}

La première horloge présente une période de $T_e + \tau$ et est dissymétrique (haut sur $\tau$ et bas sur $T_e$). La deuxième horloge est presque complémentaire. Elles ont la même durée de passage à l'état haut, mais on s'arrange pour qu'il y ait un intervalle de garde entre les moments ou le switch 1 est passant et le moment où le second est passant, sans jamais avoir les deux passants en même temps.

\begin{figure}[H]
  \centering
  \begin{circuitikz}
    \draw (0,0) node[op amp](oa){}
    (oa.-) -- ++(0,1) to[C,l=$C_2$] ++(2,0) -| (oa.out) to[short,-o] ++(1,0)node[right]{$x_e(t)$}
    (oa.+) to[spst,mirror] ++(-2,0) coordinate(A) to[C,l=$C_1$] ++(0,-2) node[ground]{}
    (A) ++ (-2,0)node[left]{$x_c(t)$} to[spst,o-] (A)
  ;\end{circuitikz}
  \caption{Schéma électrique d'un échantilloneur bloqueur}
\end{figure}


On suppose $R_{on}C << T_e,\tau_1,\tau_2$. En effet, $R_{on}$ peut être minimisé en diminuant la longueur de la grille $L_G$ de façon à ce que la charge/décharge de C soit considérée comme instantanée par rapport aux autres temps caractéristiques des signaux. On suppose également pour commencer que $x_c(t)$ évolue très lentement par rapport à la période d'échantillonnage $T_e$.\\

De $t=nT_e$ à $t=nT_e + \tau$, $\left\{ \begin{matrix}\text{switch1 passant}\\\text{switch2 bloqué}\end{matrix}\right.\rightarrow \left \{ \begin{matrix} x_E = x_c(nT_e)\\+Q = Cx_c(nT_e) = +Q_1\end{matrix} \right.$\\

De $t=nT_e + \tau$ à $t=nT_e + \tau2 - \frac{\tau_1}{2}$, les deux switchs sont bloqués, la charge +Q n'évolue pas et reste égale à $+Q_1$.\\

De $t=nT_e + \tau_2 - \frac{\tau_1}{2}$ à $t=nT_e + \tau_2 + \frac{\tau_1}{2}$, $\left\{ \begin{matrix}\text{switch1 bloqué}\\\text{switch2 passant}\end{matrix}\right.\rightarrow \left \{ \begin{matrix} x_E = v(nT_e)\\+Q = Cv(nT_e) = +Q_2\end{matrix} \right.$\\

Puis de $t=nT_e + \tau_2 + \frac{\tau_1}{2}$ à $t = (n+1)T_e$, les deux switchs sont bloqués, +Q reste égale à $+Q_2$.\\

Globalement sur une période $T_e$, on effectue un transfert de charges $\Delta Q$ à travers les deux interrupteurs, imposé par les tensions $x_c(nT_e)$ et $v(nT_e)$.\\

On a alors $\Delta Q = C(v(nT_e) - x_c(nT_e))$ sur le temps $T_e$, ce qui correspond au courant échangé via la cellule de base :
\[I = \frac{\Delta Q}{T_e} = \frac{C}{T_e}((v(nT_e) - x_c(nT_e))\]
On a une équivalence avec une résistance $R_e = \frac{1}{CF_e}$ à condition que $x_c$ et$v$ évoluent suffisamment lentement par rapport à $T_e$.\\

%\img{0.5}{2/10.png}

La valeur de $R_e$ est contrôlée par la fréquence d'échantillonnage de $F_e$. A la base de "filtres programmables" c'est à dire dont es caractéristiques peuvent être modifiées par $F_e$.\\

\section{Exemple de l'intégrateur}

Si $u$ et $v$ sont assez lents par rapport à $T_e$ et de type sinusoïdal, que l'on a un amplificateur opérationnel parfait, on a $\frac{U(j\omega)}{R_e} = -j\omega C_2V(j\omega)$, donc la fonction de transfert est:
\[\frac{V(j\omega)}{U(j\omega)} = -\frac{1}{j\omega C_2R_e} = -\frac{1}{j\omega}\frac{C_1F_e}{C_2}\]
L'intérêt par rapport à un circuit avec une "vraie" résistance $R_e$, est que la fonction de transfert dépend d'un rapport de capacités $\frac{C_1}{C_2}$ et non plus de la valeur de $C_2$ seule.\\

Remarque: $C_1$ et $C_2$ sont des capacités MOS.

%\img{0.5}{2/9.png}

Ce sont des condensateurs planaires de capacité $C=\frac{\epsilon_0 \epsilon_{ox} }{e_{ox}}WL$ où $\epsilon_0$ est la permittivité du vide et vaut $8.85\times 10^{-12}$F/m et $\epsilon_{ox}$ la permittivité relative de l'oxyde (3.8 pour du $SiO_2$).\\
Mais pour le $SiO_2$ le matériau est amorphe quand il est obtenu par oxydation thermique de Si, tandis que pour le Si, le matériau est cristallin c'est à dire que les atomes de Si sont répartis périodiquement dans l'espace.\\
Cependant, l'interface est mal définie, "rugueuse" entre les deux et donc l'épaisseur d'oxyde fluctue sur la surface WL.\\
La valeur de $C_2=\frac{\epsilon_0 \epsilon_{ox} }{e_{ox}}(WL)_2$ et de $C_1=\frac{\epsilon_0 \epsilon_{ox} }{e_{ox}}(WL)_1$ ne sont pas garantie. Mais le rapport $\frac{C_2}{C_1} = \frac{(WL)_2}{(WL)_1}$ est beaucoup mieux contrôlé.\\
Voir TP1 pour traitement plus précis de cet intégrateur...\\


Attention, le système est instable, il intègre son entrée mais aussi les défauts de l'amplificateur opérationnel dont des tensions continues de décalage, ce qui conduit à la saturation rapide de l'AO.\\
La solution est de mettre une résistance $R_2$ de grande valeur en parallèle de $C_2$, on a un gain fini pour $f<< \frac{1}{2\pi R_2C_2}$. Cette solution est difficilement intégrable.\\

 On peut aussi mettre une contre réaction par un AO câblé en soustracteur.\\

Ce sont les structures avec soustracteur qui sont à la base de "filtres universels programmables", c'est à dire d'un type de filtrage différent suivant la sortie considérée, et de fréquences caractéristiques modifiable par $F_e$.\\

\section{Exemple de filtre passe bas}

On reprend la cellule de commutation de la figure \ref{fig:commut}

Sur une période :
\begin{align*}
\intertext{Quand $H_1$ est passant et $H_2$ bloqué:	}
x(nT_e) &= u(nT_e)\\
Q_1 &= C_1 u(nT_e)\\
v(nT_e) &= w((n-1)T_e)\\
Q_2 &= C_2 w((n-1)T_e)\\
\intertext{Quand $H_2$ est passant et $H_1$ bloqué:	}
x((n+\frac{1}{2})T_e) &= v((n+\frac{1}{2})T_e) = w(nT_e)\\
Q_1 &= C_1 w(nT_e)\\
Q_2 &= C_2 w(nT_e)
\end{align*}
On effectue une re-répartition des charges présentes sur $C_1$ et $C_2$ pendant la première moitié de la période, mais on a conservation de la charge totale :
\[C_1u(nT_e) + C_2w((n-1)T_e) = (C_1 + C_2)w(nT_e)\]
C'est une "équation aux différences" liant l'entrée et la sortie du filtre.\\

En écriture simplifiée on a:
\[w_n = \frac{C_2}{C_1 + C_2}w_{n-1} + \frac{C_1}{C_1 + C_2}u_n\]

Après passage à la transformée en z on a la fonction de transfert du filtre:
\[\frac{W(z)}{U(z)} = \frac{C_1}{C_1+C_2-C_2z^{-1}} \quad \text{où } z= exp(pT_e)\]
soit:
\[H(j\omega) = \frac{C_1}{C_1+C_2-C_2exp(-j\omega T_e)}\]
Pour $f<<F_e$, $\overline{\omega} << 1$:
 \[H(j\overline{\omega}) = \frac{C_1}{C_1+C_2-C_2(1-j\overline{\omega})} = \frac{C_1}{C_1+jC_2\overline{\omega}}\]
C'est une fonction $F_e$-périodique et un filtre passe-bas pour $f<<F_e$.\\

En fait,


\begin{figure}[H]
  \centering
  \begin{tikzpicture}
    \begin{axis}
      [axis lines =middle,
      xmin = 0 ,xmax = 15,ymin = 0,ymax=0.7,height=5cm,width=10cm,
      ytick = \empty,
      xtick = {0,3,6,9,12},
      xticklabels ={0,$\pi$,$2\pi$,$3\pi$,$4\pi$},
      xlabel=$\omega$,ylabel=$|H|$]
      \addplot[no marks,black,smooth] plot coordinates
      {(0,0.5) (1,0.2) (5,0.2) (6,0.5) (7,0.2)(11,0.2) (12,0.5) (13,0.2)(17,0.2)};

    \end{axis}
  \end{tikzpicture}
  \caption{Spectre de $|H|$}
\end{figure}
Le gabarit de filtre n'est pas forcement très satisfaisant. Pour faire mieux, on utilise des filtres numériques avec une conception de filtres par rapport à un cahier des charges donné et des calculs réalisés sur circuit numériques CMOS.

\end{document}

%%% Local Variables:
%%% mode: latex
%%% TeX-master: "main"
%%% End:
