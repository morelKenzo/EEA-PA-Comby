\documentclass[main.tex]{subfiles}
\begin{document}
\section{Codage de donnée discrètes}
\begin{defin}
Les données discrètes sont représentées par des symboles en nombre fini $m$.
On parle d'une répresentation $m-$aire ou $m-$moments
\end{defin}

\begin{exemple}
  \begin{itemize}
  \item Alphabets
  \item Symbole de numérotation (décimal, hexa, octal)
  \end{itemize}
\end{exemple}
\begin{center}
\begin{tabular}{|c|c|c|c|}
  \hline
  Sources          & Symboles & Dimension        & Codage binaires \\
  \hline
  alpha. simplifié & lettre   & 27               & 5               \\
  alphabet         & lettres  & 128              & 7               \\
  Nombres          & chiffres & Dec: 0-9 10      & 4 (DCB)         \\
  Nombres          & chiffres & Hex: 0-F 16      & 4               \\
  Nombres          & chiffres & Ternaire: 0-p 10 & 2)              \\
  \hline
\end{tabular}
\end{center}

\begin{rem}
Les symboles binaire s sont des bits ou ``digit''.

On code  un alphabet à $m= 2^n$ symboles avec des mots binaires à $n$ bits. Il y a $m!$ possibilités.
\end{rem}

\section{Codage d'une information analogique MIC}
\subsection{Conversion analogique numérique}
On réalise une conversion Analogique-Numérique classique : Échantillonnage et blocage. Comme au chapitre 1.
\subsection{Bruit de quantification}
\subsection{Quantification uniforme}
Ensuite on effectue une quantification uniforme , commme au chapitre 1.
\subsection{Quantification non uniforme}

\subsection{Loi $A$ et loi $\mu$}
\paragraph{Objectif}
Rendre le rapport signal sur bruit de quantification indépendant du niveau du signal.

\begin{defin}
  On rapppelle la définition de \emph{puissance d'un signal}
  \[
    P_x=\int_{-1}^{1}x^2 p(x)\d x
  \]
  Soit pour le bruit issue d'une quantification non uniforme
  \[
    \sigma_q^2 = \int_{-1}^{1}p(x)\frac{\Delta_i^2}{12} = \int_{-1}^{1}\frac{p(x)}{12} \left(\frac{2}{N}\deriv[x]{y}\right)^2\d x
  \]
\end{defin}
\begin{prop}
  Le RSB s'écrit alors:
  \[
    \frac{P_x}{\sigma_q} = \frac{\int_{-1}^{1}x^2 p(x)\d x}{\int_{-1}^{1}\frac{p(x)}{3N^2} \left(\frac{2}{N}\deriv[x]{y}\right)^2\d x} =Cste
  \]
  Cela est possible pour $\deriv[x]{y}=kx$ soit :
  \[
\frac{P_x}{\sigma_q} = \frac{3N^2}{k^2}
\]
Soit en dB :
\[
\left(\frac{P_x}{\sigma_q}\right)= 6 n +4.7 -20 \log_{10}(k)
\]
\end{prop}


\begin{rem}
  On a alors:
   \[
     y = \frac{1}{k}\ln |x|+1
   \]
   Pour $x\simeq 0$ on doit faire une approximation.
 \end{rem}
 \begin{prop}
   \begin{itemize}
   \item Loi $\mu$ (USA)
   \[
     \begin{cases}
       y =\frac{\ln(1+\mu|x|)}{\ln(1+\mu)} \\
       \mu = 255
     \end{cases}
   \]
 \item Loi $A$ (UE)
   \[
   \begin{cases}
     y = \frac{Ax}{1+\ln(A)}  & \text{si} |x| < 1/A\\
     y= \frac{1+\ln(A|x|)}{1+\ln(A)} & \text{si} |x| \ge 1/A
   \end{cases}
   \]
 \end{itemize}
 \end{prop}
\section{Modulation différentielles DPCM}

\end{document}

%%% Local Variables:
%%% mode: latex
%%% TeX-master: "main"
%%% End:
