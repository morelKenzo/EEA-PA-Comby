\documentclass[../main.tex]{subfiles}

\begin{document}

\section*{Exercice 1 :}
Remarque : Avec ces notations on a : $i_q = -C\frac{dv}{dt}$

Lois des noeuds :\\
\begin{itemize}
\item $i = i_1 + i_2$
\item $i = i_3 + i_4$
\item $i_1 = i_q + i_3$
\item $i_4 = i_q + i_2$
\end{itemize}

Lois des mailles :\\
\begin{itemize}
\item $R_1i_1 - R_2i_2 - v = 0$
\item $R_3i_3 + v -R_4i_4 = 0$
\end{itemize}

\begin{align*}
R_1i_q + R_1i_3 - R_2i_2 - v = 0\\
R_3i_3 + v -R_4i_2- R_4i_q = 0
\intertext{ou encore :}
-R_1C\frac{dv}{dt} - v + R_1i_3 - R_2i_2 = 0\\
R_4C\frac{dv}{dt} + v R_3i_3 -R_4i_2 = 0
\end{align*}

Posons $\Delta = R_1R_4-R_2R_3$
On a donc :
\begin{align*}
i_3 &= \frac{1}{\Delta}(-(r_1+r_2)R_4C\dot{v} - (R_2+R_4)v)\\
i_2 &= \frac{1}{\Delta}(-(R_3+R_4)R_1C\dot{v}-(R_1+R_3)v)
\intertext{on a donc :}
u &= L\frac{di}{dt}+R_1i_1+R_3i_3\\
&= L\frac{di}{dt} - (R_1 + R_3)i_3 +R_1i_q\\
& \vdots\\
u &= L\frac{di}{dt}-\frac{(R_1R_2R_3+R_1R_2R_4+R_1R_3R_4+R_2R_3R_4)C}{\Delta}\dot{v} - \frac{(R_1+R_3)(R_2+R_4)}{\Delta}v
\intertext{Or, $i = i_1 + i_2$ donc $i = i_q + i_3 + i_2$, donc :}
i &= -C\dot{v} - \frac{R_1+R_2+R_3+R_4}{\Delta}v - \frac{(R_1R_3+2R_1R_4+R_2R_4)C}{\delta}\dot{v}\\
&\vdots\\
i &= -\frac{(R_1+R_2)(R_3+R_4)}{\Delta}\dot{v} - \frac{R_1+R_2+R_3+R_4}{\Delta}v
\end{align*}

On pose :
\begin{itemize}
\item $\alpha_1 = R_1R_2R_3+R_1R_2R_4+R_1R_3R_4+R_2R_3R_4$
\item $ \alpha_2 = (R_1+R_2)(R_3+R_4)$
\item $\beta_1 = (R_1+R_3)(R_2+R_4)$
\item $\beta_2 = R_1+R_2+R_3+R_4$
\end{itemize}
\bigbreak
On a alors :
\begin{align*}
u = L\frac{di}{dt} - \frac{\alpha_1C}{\Delta} \dot{v} - \frac{\beta_1}{\Delta}v\\
i = -\frac{\alpha_2C}{\Delta}\dot{v} - \frac{\beta_2}{\Delta}v
\end{align*}

Posons $x_1 = i$ et $x_2 = v$ :
\begin{align*}
u = L\dot{x_1} - \frac{\alpha_1C}{\Delta} \dot{x_2} - \frac{\beta_1}{\Delta}x_2\\
x_1 = -\frac{\alpha_2C}{\Delta}\dot{x_2} - \frac{\beta_2}{\Delta}x_2
\end{align*}



On a donc le système matriciel suivant :
\begin{align*}
\begin{pmatrix}
L & -\frac{\alpha_1 L}{\Delta} \\
0 & -\frac{\alpha_2 L}{\Delta}
\end{pmatrix}.\begin{pmatrix}
\dot{x}_1 \\
\dot{x}_2
\end{pmatrix} &= \begin{pmatrix}
0 & \frac{\beta_1}{\Delta}\\
1 & \frac{\beta_2}{\Delta}
\end{pmatrix}.\begin{pmatrix}
x_1\\
x_2
\end{pmatrix}+\begin{pmatrix}
1\\
0
\end{pmatrix}u
\intertext{On en déduit alors pour $\frac{\alpha_2CL}{\Delta} \neq 0$ l'équation d'état:}
\begin{pmatrix}
\dot{x}_1\\
\dot{x}_2
\end{pmatrix} &= \begin{pmatrix}
-\frac{\alpha_1}{\alpha_2L} & \frac{\beta_1}{\Delta}- \frac{\alpha_1\beta_2}{\alpha_2L\Delta}\\
-\frac{\Delta}{\alpha_2C} & -\frac{\beta_2}{\alpha_2}
\end{pmatrix}.\begin{pmatrix}
x_1\\
x_2
\end{pmatrix} + \begin{pmatrix}
\frac{1}{L}\\
0
\end{pmatrix}u\\
\intertext{Sans oublier l'équation d'obeservation :}
y&= i = x_1 = \begin{pmatrix}
1 & 0
\end{pmatrix}.\begin{pmatrix}
x_1\\
x_2
\end{pmatrix} + 0*u
\end{align*}
\bigbreak

\noindent 2- Théorème de caractérisation de la commandabilité :\\
On considère le système suivant :
\[(S)= \left \{
   \begin{matrix}
   \dot{x} = Ax + Bu & x(0) = x_0 \in \mathbb{R}^n\\
   y = Cx + Du &
   \end{matrix}
   \right.
\]

\noindent (S) est commandable $\Leftrightarrow$ rang[C(A,B)] = n (matrice de rang plein)\\
Où C(A,B) = $\begin{pmatrix}
A^0B & AB & A^2B & ...& A^{n-1}B
\end{pmatrix}\in \mathbb{R}^{n*n}$\\

Remarque : on est en monovariable $\Leftrightarrow$ $det (C(A,B)) \neq 0$\\

On calcul donc :
\begin{align*}
A &= \begin{pmatrix}
-\frac{\alpha_1}{\alpha_2L} & \frac{\beta_1}{\Delta}- \frac{\alpha_1\beta_2}{\alpha_2L\Delta}\\
-\frac{\Delta}{\alpha_2C} & -\frac{\beta_2}{\alpha_2}
\end{pmatrix}\\
A^0B &= \begin{pmatrix}
\frac{1}{L}\\
0
\end{pmatrix}\\
AB &= \begin{pmatrix}
-\frac{\alpha_1}{\alpha_2L^2}\\
-\frac{\Delta}{\alpha_2LC}
\end{pmatrix}
\intertext{On a donc :}
C(A,B) &= \begin{pmatrix}
\frac{1}{L} & -\frac{\alpha_1}{\alpha_2L^2}\\
0 & -\frac{\Delta}{\alpha_2LC}
\end{pmatrix}
\intertext{On calcul alors le determinant :}
det(C(A,B)) = \frac{-\Delta}{\alpha_2L^2C} \neq 0 &\Leftrightarrow \Delta \neq 0 \\ &\Leftrightarrow R_1R_4 - R_2R_3 \neq 0
\end{align*}
\bigbreak
Remarque : si $R_1R_4 = R_2R_3$ le pont est équilibré et $v(t) = 0$ donc le système est non commandable.


\section*{Exercice 2 :}
\noindent 1-a) \begin{align*}
P_a(\lambda) &= det\begin{pmatrix}
-1-\lambda & 1 & 0\\
-1 & -3-\lambda & 0\\
1 & 1 & -2-\lambda
\end{pmatrix}\\
&= (-1-\lambda)(-3-\lambda)(-2-\lambda) + (-2-\lambda)\\
&= (-2-\lambda)(+3+\lambda +3\lambda + \lambda^2+1)\\
&= (-2-\lambda)^3
\end{align*}

\noindent 1-b) $\lambda_0 = -2$ vecteur propre triple\\

\noindent 1-c) Cherchons les vecteurs propres vérifiant $AX = \lambda X$:
\begin{align*}
\begin{matrix}
-x_1 + x_2 &= -2x_1\\
-x1 -3x_2 &= -2x_2\\
x_1+x_2 - 3x_3 &= -2x_3
\end{matrix} &\rightarrow x_1 + x_2 =0
\end{align*}
On choisit donc $x_3$ quelconque et $x_1=-x_2$ ce qui correspond à un sous espace propre de dimension 2 et :
\begin{align*}
Ker\{ \lambda_0 \mathbb{1}_3 - A\} &= Vect \left\{ \begin{pmatrix}
1\\-1\\0
\end{pmatrix} , \begin{pmatrix}
0\\0\\1
\end{pmatrix} \right \}
\end{align*}
La dimension de ce sous espace propre est $2 \leq 3$ donc A est non diagonalisable.\\
On a donc deux blocs de Jordan car la multiplicité des valeurs propres de 2 :
\[J=\begin{pmatrix}
\lambda_0 & 1 & 0\\
0 & \lambda_0 & 0\\
0 & 0 & \lambda_0
\end{pmatrix} \textbf{ 	ou alors   } \begin{pmatrix}
\lambda_0 & 0 & 0\\
0 & \lambda_0 & 1\\
0 & 0 &\lambda_0
\end{pmatrix} = J\]

Le but maintenant est de trouver un 3ème vecteur pour compléter la bases de vecteurs propres et avoir $V = \begin{pmatrix}v_1 & v_2 & v_3 \end{pmatrix}$ tel que $V^{-1}AV = J$, avec J l'une des deux matrices contenant un bloc de Jordan.\\
En posant AV = VJ on a :
\begin{align*}
\begin{pmatrix}
Av_1 & Av_2 & Av_3
\end{pmatrix}&= \begin{pmatrix}
\lambda_0v_1 & \lambda_0 v_2 & v_2 + \lambda_0 v_3
\end{pmatrix}
\intertext{et on obtient le système: }
\begin{matrix}
Av_1\\
Av_2\\
Av_3
\end{matrix} &= \begin{matrix}
\lambda_0v_1\\
\lambda_0 v_2\\
v_2 + \lambda_0 v_3
\end{matrix}
\intertext{on obtient donc en particulier pour le vecteur $v_3$ :}
\begin{pmatrix}
1 & 1& 0\\
-1& -1 & 0\\
1& 1& 0
\end{pmatrix} v_3 = \begin{pmatrix}
0\\0\\1
\end{pmatrix}
\end{align*}
On constate qu'il n'est pas possible de déterminer $v_3$ de cette façon, on cherche donc a prendreà la place de $v_2$, $\tilde{v_2} = \alpha v_1 + \beta v_2$ de sorte à avoir $A\tilde{v_2} = \lambda_0 \tilde{v_2}$ (ce qui reste vrai car on prend une combinaison liéaire de 2 vecteur propre du ker) et surtout $(A-\lambda_0 \mathbb{1}_3)v_3 = \tilde{v_2}$. On a donc :
\begin{align*}
\begin{pmatrix}
1 & 1& 0\\
-1& -1 & 0\\
1& 1& 0
\end{pmatrix} v_3 &= \alpha \begin{pmatrix}
1\\-1\\0
\end{pmatrix} + \beta \begin{pmatrix}
0\\0\\1
\end{pmatrix}
\intertext{avec , $\beta = \alpha = 1$}
\tilde{v_2} &= \begin{pmatrix}1\\-1\\1\end{pmatrix}
\intertext{on résoud alors pour trouver $v_3$ :}
v_3 = \begin{pmatrix}0\\1\\1\end{pmatrix}
\intertext{on a donc finalement trouvé V :}
V &= \begin{pmatrix}
1&1&0\\-1&-1&1\\0&1&1
\end{pmatrix}\\
V^{-1} &= \begin{pmatrix}
2&1&-1\\-1&-1&1\\1&1&0
\end{pmatrix}\\
J &= V^{-1}AV = \begin{pmatrix}
-2&0&0\\0&-2&1\\0&0&-2
\end{pmatrix}
\end{align*}
On a bien une matrice diagonale par bloc avec un bloc de Jordan.
On pose $\xi \in \mathbb{R}^3$, $x = V\xi$, et on a le système :
\begin{align*}
\left \{ \begin{matrix}
\dot{x} = Ax + Bu & x_0 \in \mathbb{R}^n\\
y = Cx + Du &
\end{matrix} \right. &\rightarrow \left \{ \begin{matrix}
\dot{\xi} = J\xi + V^{-1}B u &= J\xi + \tilde{B}u\\
y = CV \xi +D u &= \tilde{C}\xi + \tilde{D} u
\end{matrix} \right.
\intertext{avec, }
\tilde{B} &= \begin{pmatrix}-1\\1\\1\end{pmatrix}\\
\tilde{C} &= \begin{pmatrix}0 &1&0\end{pmatrix}\\
\tilde{D} &= 0
\end{align*}
\bigbreak
\noindent 2- \begin{align*}
\dot{\xi} &= \begin{pmatrix}-2&0&0\\0&-2&1\\0&0&-2\end{pmatrix}\xi + \begin{pmatrix}-1&1&1\end{pmatrix}\\
y &= \begin{pmatrix}0&1&0\end{pmatrix} \xi\\
C(A,B) &= \begin{pmatrix}0&1&-4\\1&-3&8\\2&-3&4\end{pmatrix}\\
det(C(A,B)) &= 0
\end{align*}
Non commandable.
\end{document}
