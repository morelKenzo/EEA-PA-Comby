% Relu 12.10.14. AA

\documentclass[../main.tex]{subfiles}

\begin{document}
\subsection*{Exercice I : Stabilité d'un asservissement avec retour unitaire}

On considère l'asservissement analogique où :
\[ H(p) = \frac{C}{p(1+\tau p)} \text{avec } \tau = 0.2 \]

On place un CNA (BOZ) en amont de $H(p)$ et un CAN dans la boucle de retour.
\begin{enumerate}
\item Pour passer de l'analogique au numérique, on utilise la formule suivante :
\[ T(z) = (1-z^{-1})Z[^*L^{-1}[\frac{H(p)}{p}]] \]
On a donc successivement :
\begin{align*}
A(p) & = \frac{H(p)}{p}  = \frac{C/\tau}{p^2(p+1/\tau)} \\
& = \frac{C}{\tau}(\frac{\alpha}{p} + \frac{\beta}{p^2} + \frac{\gamma}{p+1/\tau}) \\
& = C (\frac{-\tau}{p} + \frac{1}{p^2} + \frac{\tau}{p+1/\tau}) \\
a(t) & = C[-\tau+t+\tau e^{-\frac{t}{\tau}}] \\
a_k & = C[-\tau + kT_e + \tau e^{-\frac{T_e}{\tau}k}] \\
A(z) & = C[ -\frac{\tau z}{z-1} + \frac{T_e z}{(z-1)^2} + \tau\frac{z}{z-D} ] \text{ où } D = e^{-\frac{T_e}{\tau}} \\
T(z) & = (1-z^{-1})A(z) \\
& = \frac{z-1}{z} C[ -\frac{\tau z}{z-1} + \frac{T_e z}{(z-1)^2} + \tau\frac{z}{z-D}] \\
& = C[ -\tau + \frac{T_e}{z-1} + \frac{\tau(z-1)}{z-D} ] \\
T(z) & = C \frac{(\tau(1+D)+T_e-2\tau)z + (-D\tau - T_e D + \tau)}{z^2 - (1+D)z + D}
\end{align*}
On pose
\begin{align*}
T(z) & = \frac{b_1 z + b_0}{z^2 + a_1 z + a_0} \\
& b_1 = C(\tau(D-1) + T_e)\\
& b_0 = C(\tau(1-D) - T_e D)\\
& a_1 = -(1+D)\\
& a_0 =D
\end{align*}

\item Mise en équation de l'asservissement
\begin{align*}
Y(z) & = \frac{T(z)}{1+T(z)} E(z) \\
& = \frac{B(z)}{A(z) + B(z)} \text{ avec } T(z) = \frac{A(z)}{B(z)}
\end{align*}

Le polynôme caractéristique s'écrit :
\begin{eqnarray*}
\Pi(z) & = &  B(z) + A(z)  \\
& = &  c_2 z^2 + c_1 z + c_0 \\
& \text{avec } & c_2 = 1 \\
& & c_1 = a_1 + b_1  = -(1+D) + C(T_e + \tau (D-1)) \\
& & c_1 = -(1+D) + CT_eD \text{ car ici } \tau = T_e \\
& & c_0 = a_0 + b_0 = D + C(\tau(1-D) - T_e D) \\
& & c_0 = D + CT_e(1-2D)
\end{eqnarray*}


\item a) Critère de Routh-Hurwitz \\
Transformation en w : \( z = \frac{1+w}{1-w}, \quad w = \frac{z-1}{z+1} \) \\

On réécrit l'équation caractéristique en $w$ :
\begin{eqnarray*}
c_2 (\frac{1+w}{1-w})^2 & + c_1(\frac{1+w}{1-w}) & + c_0  = 0 \\
C_2 (w^2 + 2w + 1) & + c_1(1-w^2) & + c_0(w^2 -2w+1)  = 0 \\
(c_2-c_1+c_0)w^2 & + 2(c_2-c_0)w & + (c_2+c_1+c_0)  = 0
\end{eqnarray*}

Tableau de Routh :\\
\begin{figure}[h!]
\centering
\begin{tabular}{|c|c|c|}
\hline
$w^2$ & $c_2-c_1+c_0$ & $c_2+c_1+c_0$ \\
\hline
$w$ & $2(c_2-c_0)$ & 0 \\
\hline
$w^0$ & $c_2+c_1+c_0$ & \\
\hline
\end{tabular}

\end{figure}


On doit avoir tous les termes de la première colonne positifs : on retrouve le critère de Jury.



\item b) Critère de Jury :
\begin{itemize}
\item $c_2 + c_1 + c_0 > 0$
\item $c_2 - c_1 + c_0 > 0$
\item $|c_0| < c_2 \Leftrightarrow -c_2 < c_0 < c_2$
\end{itemize}

On traduit ces conditions
\begin{itemize}
\item $c_2 + c_1 + c_0 > 0$
\begin{align*}
& 1 + (-(1+D) + CT_eD) + (D + CT_e(1-2D)) > 0 \\
& C ( T_e D + T_e - 2 DT_e ) > 0 \\
& C (1-D) > 0 \\
& C > 0 \text{ car } D =e^{-1} < 1
\end{align*}

\item $c_2 - c_1 + c_0 > 0$
\begin{align*}
& 1 - (-(1+D) + CT_eD) + (D + CT_e(1-2D)) > 0 \\
& 2 + 2D + cT_e(-3D+1) > 0 \\
& C T_e(1-3D) > -2 -2D  \\
& C < -\frac{2+2D}{1-3D} \text{  car } 3D > 1
\end{align*}

\item $-c_2 < c_0 < c_2$
\begin{eqnarray*}
-1 < & D + CT_e(1-2D) & < 1 \\
-1 - D < & CT_e(1-2D) & < 1 - D \\
\frac{-1-D}{T_e(1-2D)} < & C & < \frac{1-D}{T_e(1-2D)}
\end{eqnarray*}
\end{itemize}

Le critère de Jury aboutit donc à :
\[ \boxed{ 0 < C < \min(-\frac{2+2D}{(1-3D)T_e},\frac{1-D}{(1-2D)T_e}) } \]

\item On s'intéresse à l'asservissement analogique du même système
\[ Y(p) = \frac{H(p)}{1+H(p)} E(p) \]
L'équation caractéristique conduit à
\begin{align*}
1 + H(p) & = 0 \\
1 + \frac{C}{p(1+\tau p)} &= 0 \\
\tau p^2 + p + C &= 0
\end{align*}
Le système est stable si et seulement si $C>0$.\\
En analogique, la marge de gain est infinie (la phase n'est jamais égale à $-180^o$. En numérique, le BOZ induit un déphasage qui conduit à une limitation supplémentaire pour $C$ en terme de stabilité.


\item Pour approcher le comportement basse fréquence de la chaîne directe de l'asservissement de la figure 1, il faut tenir compte du BOZ. Il faut approximer l'expression de $B_0(p)$.

\begin{itemize}
\item Avec $\omega << \frac{1}{T_e}$, $e^{-T_ep} \approx 1 - T_ep$ et
\begin{align*}
B_0(p) & = T_e \\
\tilde{H}(p) & = T_e H(p) = \frac{CT_e}{p(1+\tau p)}
\end{align*}
On revient à la même condition $ C > 0 $.

\item Suggestion : faire le développement à l'ordre 2 ?

Avec $e^{-T_e p} \approx 1 - T_ep + \frac{(T_ep)^2}{2}$,
\begin{align*}
B_0(p) & = \frac{T_ep-T_e^2p^2 / 2 }{p}\\
& = T_e - \frac{T_e^2}{2}p \text{ : non causal} \\
\tilde{H}(p) & = T_e(1-\frac{T_e}{2}p)\frac{C}{p(1+\tau p)}
\end{align*}
L'équation caractéristique $1+\tilde{H}(p) = 0$ mène à
\begin{align*}
\tau p^2 + p + CT_e(1-\frac{T_e}{2}p) & = 0 \\
\tau p^2 + (1-C\frac{T_e^2}{2})p + CT_e & = 0
\end{align*}
L'application du critère de Routh mène à : $0 < C < \frac{2}{T_e^2}$

\item Approximation de Padé ($\omega << \frac{1}{2T_e}$) :
\[ e^{-T_ep} = \frac{e^{-\frac{T_e}{2}p}}{e^{\frac{T_e}{2}p}} \text{ et } e^{-\frac{T_e}{2}p} \approx 1 -
\frac{T_e}{2}p \Longrightarrow e^{-T_ep} \approx \frac{1 - \frac{T_e}{2}p}{1 + \frac{T_e}{2}p} \text{ : causal !} \]

\begin{align*}
B_0(p) & \approx \frac{1-\frac{1 - \frac{T_e}{2}p}{1 + \frac{T_e}{2}p}}{p} \\
& = \frac{T_e}{1+\frac{T_e}{2}p} \text{ : causal } \\
\tilde{H}(p) & = \frac{T_e}{1+\frac{T_e}{2}p} . \frac{C}{p(1+\tau p)}
\end{align*}

L'équation caractéristique $1+\tilde{H}(p)=0$ mène à
\begin{align*}
CT_e + p (1+\frac{T_e}{2}p)(1+\tau p) & = 0 \\
\tau\frac{T_e}{2}p^3 + (\tau+\frac{T_e}{2})p^2 + p + CT_e &= 0
\end{align*}

\begin{figure}[h!]
\centering
\begin{tabular}{|c|c|c|}
\hline
$p^3$ & $\tau \frac{T_e}{2}$ & 1 \\
\hline
$p^2$ & $\tau + \frac{T_e}{2} $ & $CT_e$ \\
\hline
$p$ & $1- \frac{CT_e \tau}{2(\tau + \frac{T_e}{2})}$ & 0 \\
\hline
$p^0$ & $CT_e$ &  \\
\hline
\end{tabular}
\end{figure}

La condition est donc $C < \frac{2(\tau + T_e/2)}{T_e \tau} = \frac{3}{\tau}$
\end{itemize}

\item La discrétisation d'un asservissement en temps continu dégrade la stabilité.
\end{enumerate}

\end{document}
