\documentclass[../main.tex]{subfiles}

\begin{document}

\subsection*{Exercice 1 : Asservissement par synthèse d'un correcteur RST}
On considère la fonction de transfert: \[G(p) = \frac{1}{p(1+\tau p)}\]

Rappel : (TD précédent)
\begin{align*}
G(z) & = \frac{B(z)}{A(z)} \\
& = \frac{b_1z+b_0}{(z-1)(z-D)} \avec D=e^{-T_e/2}\\
& =\frac{b_1z+b_0}{z^2+a_1z+a_0}  \\
n & = deg(A) = 2 \\
m & = deg(B) = 1 \\
\end{align*}

\begin{enumerate}

\item Il y a plusieurs façon de représenter le correcteur RST :
\begin{itemize}
\item $\nu(z)=-\frac{S(z)}{R(z)}Y(z) + \frac{T(z)}{R(z)}E(z)$

Les conditions de causalité sont alors : $\tau \leq p$ et $\sigma \leq p$.\\

\item $\nu(z)=\frac{S}{R}[-Y(z) + \frac{T}{S}E(z)] $

Causalité : $\tau \leq p$ et $\sigma \leq p$.\\

\item $\nu(z)=\frac{T}{R}[-\frac{S}{T}Y(z) + E(z)]$

Causalité : $\tau \leq p$ et $\sigma \leq p$.\\
\end{itemize}


\item On a : $Y(p)= G(p)[\nu(p) + P(p)]$, $P(p) = \frac{P_0}{p}$ et $\nu(p) = B_0(p)\nu^*(p)$\\

\begin{align*}
Y(p) &=  G(p)[\nu(p) + P(p)]\\
&= G(p)[B_0(p)\nu^*(p)+\frac{P_0}{p}]\\
&= \frac{1-e^{-T_ep}}{p}G(p)\nu^*(p) + G(p)\frac{P_0}{p}\\
\intertext{On pose $A(p) = \frac{G(p)}{p}$, et on a après transformée inverse de Laplace :}
y(t) &= (a*\nu^*)(t)-(a*\nu^*)(t-T_e)+a(t)P_0\\
\intertext{après discrétisation on obtient : }
y_k & = a_k*\nu_k-a_{k-1}*\nu_{k-1} + a_kP_0\\
\intertext{après transformée en $z$ on a alors :}
Y(z) &= A(z)\nu(z)-z^{-1}A(z)\nu(z)+A(z)P_0\\
&=(1-z^{-1})A(z)\nu(z)+A(z)P_0\\
&= G(z)\nu(z)+(1-z^{-1})A(z)(1-z^{-1})^{-1}P_0\\
&= G(z)\nu(z)+G(z)P_0\frac{z}{z-1}\\
Y(z)&= G(z)(\nu(z)+p(z))
\end{align*}
\bigbreak

\item Calcul de R et S.

On considère l'exigence 1 du cahier des charges :\\
$p_{1,2}=\omega_0 (-\xi \pm j\sqrt{1-\xi^2})$, avec $\omega_0 = 1.5$ et $\xi=0.7$

$p_3 = -3\omega_0 \xi$\\

Comme $z_i = e^{T_ep_i}$, avec $\Omega = \omega_0 \sqrt{1-\xi^2}$,

$z_1 = e^{-T_e\omega_0\xi}(cos(\Omega T_e) +j sin(\Omega T_e))$

$z_2 = \overline{z_1}$

$z_3 = e^{-3\omega_0 \xi T_e}$\\

En boucle fermée, l'asservissement considéré donne en posant $G(z) = \frac{B(z)}{A(z)}$,
\[Y(z) = \frac{BT}{AR+BS}E(z) + \frac{BR}{AR+BS}p(z)\]
On pose $H_d(z) = \frac{B_d(z)}{\Pi_d(z)} = \frac{BT}{AR+BS}$, avec $deg(\Pi_d)=q$ et $deg(B_d)=\mu$. Les pôles (continus) imposés par le cahier des charges conduisent à la forme suivante pour le polynôme caractéristique :
\begin{align*}
\Pi_d(z) &= (z-z_1)(z-z_2)(z-z_3)\\
&= z^3+C_2z^2+C_1z+C_0\\
\intertext{avec}
C_0 &= z_1z_2z_1 = -e^{-5\omega_0 \xi T_e}\\
C_1 &= z_1z_2 + z_2z_3 + z_3z_1 = (e^{-2 \xi \omega_0 T_e}+2e^{-4\xi \omega_0 T_e}cos(\Omega T_e))\\
C_2 &= -z_1-z_2-z_3 = -(e^{-3\xi \omega_0 T_e}+2e^{-\xi \omega_0 T_e}cos(\Omega T_e))\\
\end{align*}



On s'intéresse à l'exigence 2 du cahier des charges :
Pour $E(z) = 0$ i.e $e_k = 0$, et $P(z) = P_0 \frac{z}{z-1}$,
\[\lim_{\rightarrow\infty} y_k =0\]

Ainsi, avec le théorème de la valeur finale on a :
\begin{align*}
\lim_{z\rightarrow 1} \frac{z-1}{z}Y(z) = \lim_{z\rightarrow 1} \frac{BR}{AR+BS}P_0 =0
\Leftrightarrow & \lim_{z\rightarrow 1} B(z)R(z) = 0\\
\Leftrightarrow & R(z) = (z-1)^l \tilde{R}(z)\text{\indent avec, } l\geq 1\\
\end{align*}

Par simplicité, on prend donc $l=1$.

\[\frac{B(z)T(z)}{A(z)R(z)+B(z)S(z)} = \frac{B_d(z)}{\Pi_d(z)}\]
donc $B(z)T(z) = B_d(z)$ et $A(z)(z-1)\tilde{R}(z)+B(z)S(z)=\Pi_d(z)$\\
donc avec $\tilde{A}(z) = (z-1)A(z)$, \[\boxed{\tilde{A}(z)\tilde{R}(z) + B(z)S(z) = \Pi_d(z)}\]


Soit $\tilde{n} = deg(\tilde{A}) = n+1$ et $\tilde{\rho} = deg(\tilde{R})$\\

1) Égalité des degrés : $deg(\tilde{A}\tilde{R})=deg(AR)<deg(BS)$ par causalité, donc
\[\tilde{n}+\tilde{\rho} = q = 3\]

2) Nombre d'inconnues = nombre d'équations : \\
$\tilde{R}$ monique $\rightarrow \tilde{\rho}$ inconnues\\
$S$ non monique $\rightarrow \sigma +1$ inconnues\\
$\Pi_d$ monique $\rightarrow q$ inconnues\\
donc on a \[\tilde{\rho}+\sigma +1 = q\]

3) Causalité du correcteur $\frac{S(z)}{R(z)}$, donc on a nécessairement $\sigma \leq \rho = \tilde{\rho}+1$
d'où on a
\[ \sigma = \tilde{n} -1 \leq \tilde{\rho} + 1 = q - \tilde{n} +1 \]

donc
\[q \geq 2\tilde{n}-2 = 4\]

Or, on avait déterminé  $q=3$, donc il faut donc introduire $A_0(z)$ un polynôme auxiliaire/observable de degrés $k$ qui doit être monique dans le numérateur et le dénominateur.
Les racines de $A_0(z)$ sont choisies plus rapide que $z_1$, $z_2$, et $z_3$ (d'au moins une décade).


On pose donc : $\frac{BT}{AR+BS} = \frac{B_d}{\Pi_d}\frac{A_0}{A_0}$ d'où :
\begin{align*}
AR+BS & = \Pi_d A_0\\
\tilde{A}\tilde{R} + BS & = \Pi_d A_0
\end{align*}
1) Égalité des degrés
\[\tilde{n} + \tilde{\rho} = q + k \text{ avec, $q=3$ donné}\]

2) Nombre d'inconnues = nombre d'équations
\[ \tilde{\rho} + \sigma +1 = q +k\]

3) Causalité
\[\tilde{\rho} \geq \sigma -1\]

Avec 1) et 2),
\[\sigma = \tilde{n}-1 = 2\]

Avec 3) et 1)
\[-\tilde{n}+q+k \geq \tilde{n}-2\]
\[k \geq 2\tilde{n}-2 -q = k_{min} = 1
\]

\textbf{Bilan :}\\
On a donc :
\[ k \geq 1, \quad \sigma = 2, \quad \tilde{\rho} \geq 1\]
Idéalement, on prend $k=1$ et la racine de $A_0$ doit être prise plus rapide (au moins une décade) que $z_1,z_2$ et $z_3$\\
\[A_0(z) = z-z_0 \avec z_0 = e^{-30\xi \omega_0 T_e}\]


On a alors :
\begin{align*}
\tilde{R}(z) &= z + r_0 \\
S(z) &= s_2z^2+s_1z+s_0\\
\tilde{\Pi_d}(z) &= \Pi_d(z)A_0(z)\\
&=(z^3+C_2z^2+C_3z+C_0)(z-z_0)\\
&= z^4+\tilde{C_3}z^3+\tilde{C_2}z^2+\tilde{C_1}z+\tilde{C_0}\\
\avec \tilde{C}_3 & = C_2 - z_0 \\
\tilde{C}_2 &= C_1 - z_0C_2 \\
\tilde{C}_1 &= C_0 - z_0C_1 \\
\tilde{C}_0 &= -z_0C_0
\intertext{et}
\tilde{A}(z) &= (z-1)A(z) = (z-1)(z^2+a_1z+a_0)\\
&= z^3 + \tilde{a_2}z^2+\tilde{a_1}z+\tilde{a_0} \\
\avec \tilde{a_2}& =a_1-1\\
\tilde{a_1}& =a_0-a_1\\
\tilde{a_0}& = -a_0
\end{align*}

L'équation $\tilde{A}\tilde{R} + BS = \Pi_dA_0$ peut donc se réécrire sous la forme matricielle suivante :

\[
\left[
\begin{array}{cccc}
1 & b_1 & 0 & 0 \\
\tilde{a}_2 & b_0 & b_1 & 0 \\
\tilde{a}_1 & 0& b_0 & b_1 \\
\tilde{a}_0 & 0 & 0 & b_0
\end{array}
\right]
\left[
\begin{array}{c}
r_0 \\
s_2 \\
s_1 \\
s_0
\end{array}
\right]
=
\left[
\begin{array}{c}
\tilde{C}_3 - \tilde{a}_2 \\
\tilde{C}_2 - \tilde{a}_1 \\
\tilde{C}_1 - \tilde{a}_0 \\
\tilde{C}_0
\end{array}
\right]
\]

La résolution de cette équation donne les polynômes
\begin{align*}
R(z) & = (z-1)(z+r_0)\\
S(z) & = s_2+z^2+s_1z+s_0
\end{align*}

Il ne reste qu'à déterminer le polynôme $T(z)$.

\[ \text{Rappel : } B(z)T(z) = B_d(z)A_0(z)\]
\[\frac{BT}{AR+BS} = \frac{B_d}{\Pi_d}\frac{A_0}{A_0}\]

On note : $\mu = deg B_d$, et on a $k = deg A_0$\\
Par causalité, \[deg(B_dA_0) = \mu +k \leq deg(\Pi_dA_0) = q + k\]
\[ \mu \leq k = 3 \]
En boucle ouverte : on a le retard $n-m = 2-1 = 1$
\[q-\mu \geq 1 \longrightarrow \mu \leq q-1 = 2\]
L'égalité des numérateurs donne alors $m+\tau = \mu + k$ donc
\[\mu = m+\tau-k = \tau\]
\[\tau \leq 2\]

Avec $B_d(z) = B(z)\tilde{B}(z)$,
\begin{align*}
BT & = B_dA_0 \\
B(z)T(z) &= B(z)\tilde{B}(z)A_0(z) \\
T(z) &= \tilde{B}(z)(z-z_0)
\end{align*}

Solution la plus simple pour $\tau \leq 2$ : $\tilde{B}(z) = 1$ donc $B_d(z) = B(z)$, et \[T(z) = A_0(z)\]

\item
\begin{align*}
Y(z) & = \frac{B_d(z)}{\Pi_d(z)}E(z) + \frac{B_p(z)}{\Pi_d(z)}P(z)
\intertext{En l'absence de perturbation, on veut que pour}
E(z) & = \frac{E_0z}{z-1}
\intertext{on ait :}
Y(z) & = \frac{E_0z T_e}{(z-1)^2}\\
\intertext{donc}
\frac{B_d(z)}{\Pi_d(z)} & = \frac{T_e}{z-1} \longrightarrow \Pi_d(z) = z-1
\end{align*}

Compensation du zéro de B(z) avec $B(z) = b_1z+b_0 = b_1(z+\frac{b_0}{b_1})$
or, $b_0 = 0.2(1-2D)$, et $b_1 = 0.2D$, donc $\frac{b_0}{b_1}= \frac{1-2D}{D}$ avec $D=e{-1}$
donc $ \frac{b_0}{b_1} = \frac{0.052848}{0.073576} < 1$ donc le zéro est stable et on peut le compenser.


$H_d(z) = \frac{T_e}{z-1}  \rightarrow $ retard $=q-\mu = 1 \geq m-n$. OK. \\

Modèle admissible :
\[\frac{BT}{AR+BS} = \frac{B_d}{\Pi_d} \frac{A_0}{A_0}\]
\begin{align*}
\lim_{n \rightarrow \infty} y_n = 0 & \Leftrightarrow \lim_{z \rightarrow 1} \frac{z-1}{z}\frac{BR}{AR+BS} p(z) = 0\\
& \Leftrightarrow \lim_{z \rightarrow 1} \frac{z-1}{z}\frac{BR}{\Pi_dA_0} p(z) = 0\\
& \Leftrightarrow \lim_{z \rightarrow 1} \frac{z-1}{z} \frac{BR}{(z-1)A_0(z)} \frac{p_0 z}{z-1} = 0
\end{align*}

Il faut, $R(z) = (z-1)^l\tilde{R}(z)$ avec $l \geq 2$.
On prend $l = 2$.

\begin{align*}
B(z) &= b_1(z+ \frac{b_0}{b_1})\\
&=B_s(z)B_{ns}(z)\\
\avec B_s(z) & = (z+\frac{b_0}{b_1})\\
\et  B_{ns}(z) & = b_1
\intertext{on a donc :}
\frac{B_sB_{ns}T}{AR+B_sB_{ns}S}&= \frac{B_dA_0}{\Pi_d A_0}
\intertext{Comme $R = (z-1)^2\tilde{R}(z)$, en posant $\tilde{R} = B_s\hat{R}$ ($\hat{R}$ monique) et $\tilde{A}(z) = (z-1)^2A(z)$ on obtient }
\frac{B_{ns}T}{\tilde{A}\hat{R}+B_{ns}S}&= \frac{B_dA_0}{\Pi_d A_0}
\intertext{D'où l'équation diophantine}
\tilde{A}\hat{R}+B_{ns}S & = \Pi_dA_0
\end{align*}

On pose $\hat{\rho} = deg(\hat{R}) $ et $\tilde{n}=deg(\tilde{A})=n+l=4$.\\

1) Égalité des degrés :
\[\tilde{n} + \hat{\rho} = q+k \]

2) Nombre d'inconnues = nombre d'équations :
\[\hat{rho} + \sigma +1 = q+k \]

3) Causalité :
\[\rho \geq \sigma \]

Avec 1) et 2),
\[\sigma = \tilde{n}-1 = 3\]

Comme $\rho = l + \hat{\rho} + m_S$,
\[ \sigma = \tilde{n}-1 \ leq l + m_S + \hat{\rho} = l + m_S + q + k - \tilde{n}\]
\[ \rightarrow k \geq k_{min} = 3 \]

On choisit $k=3$, et on a alors $\hat{\rho}=0$.

Par conséquent, on a $\hat{R}(z) = 1$ et on prend $S(z) = s_3z^3 + s_2z^2 + s_1z^1 + s_0$.

L'équation se ramène donc à
\[\tilde{A}+B_{nS}S = \Pi_dA_0\]
\[S = \frac{1}{b_1}(\Pi_dA_0 - \tilde{A}) \]

Comme $\Pi_dA_0 = (z-1)A_0$ avec $A_0=(z-z_1)(z-z_2)(z-z_3)=z^3+C_2z^2+C_1z+C_0$
\[\Pi_dA_0 = z^4 + \gamma_3z^3 + \gamma_2z^2 + \gamma_1z + \gamma_0\]
avec $\gamma_3 = c_2-1, \gamma_2 = c_1-c_2, \gamma_1 = c_0-c_1, \gamma_0=-c_0$.\\

De plus, $\tilde{A}(z) = (z-1)^2 A(z) = z^4 + \tilde{a}_3z^3 + \tilde{a}_2z^2 + \tilde{a}_1z^1 + \tilde{a}_0$\\
avec $\tilde{a}_3 = a_1 -2, \tilde{a}_2 = 1 -2a_1+a_0, \tilde{a}_1 = a_1-2a_0, \tilde{a}_0 = a_0$\\

On a donc, \[\forall j =0,...,3, \quad s_j=\frac{\gamma_j-\tilde{a}_j}{b_1}\]

Il reste donc à déterminer le polynôme $T(z)$.


\end{enumerate}

\end{document}
