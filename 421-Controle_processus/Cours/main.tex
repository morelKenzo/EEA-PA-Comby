\documentclass{../../cours}
\usepackage{../../raccourcis}
\usepackage{multicol}
% Mise en page
\title{Contrôle de processus}
\author{Pierre-Antoine Comby}
\teacher{Samy Tliba}
\module{421}

\begin{document}

\maketitle
\tableofcontents


\chapter{Echantillonnage des signaux et transformée en z}
\subfile{chap1.tex}
\chapter{Fonctions de transfert en z}
\subfile{chap2.tex}

\chapter{Correcteur RST}
\subfile{chap3.tex}
\newpage
\textbf{À savoir pour le partiel}
\begin{itemize}
\item Transformées en $z$ et ses propriétés
\item Tous les résultats sur les filtres linéaires numériques
\item Méthode d'obtention des transformées en $z$ d'un système discrétisé
\item Définitions stabilité, critères algébriques pour la caractériser (connaître Routh-Hurwitz, savoir appliquer Jury)
\item Réglage d'un correcteur PID analogique, savoir passer en TD (Euler, Tustin, méthode de correspondance pour le zéro)
\item Transformation en $w$
\item Correcteur RST
\item Règle des retards relatifs
\item Règles de rejet de la perturbation, conséquence sur le polynôme $R$
\item Être en mesure de déterminer les degrés de polynômes R et S pour résoudre un problème : technique de simplification de pôles / zéros, introduction de polynôme auxiliaire
\end{itemize}

\chapter{Commande dans l'espace d'état}
\subfile{chap4.tex}
\end{document}
