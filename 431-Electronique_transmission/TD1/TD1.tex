\documentclass[../../Cours_M1.tex]{subfiles}
\newcommand{\nomTD}{TD1 : L'amplificateur opérationnel}
\renewcommand{\nomentete}{UE431 - \nomTD}

\begin{document}

\section*{\nomTD}

\subsection*{Exercice 1}
Le but de ce TD est de valider les hypothèse $\mu \neq \infty$ et $R_e \neq \infty$ et $R_0=0$.\\
On considère le montage suivant :\\
\medbreak
\begin{center}
\begin{tikzpicture}[scale = 1]
  \draw[color=black]
(1,1) node[op amp] (opamp) {}
(opamp.+) node[left] {}
(opamp.-) node[left] {}
(opamp.out) node[right] {}
(2,-1) to[open,v_>=$V_s$] (2,1)
(-1,1.5)--(opamp.-)--(-0.5,1.5)--(-0.5,2.5)to[R, l=$R_2$,i<=$i_2$](2,2.5)--(2,1)
(-1,0.5)--(opamp.+)
(-1,0.5)--(-1,-.2)to[V,v^<=$V_e$](-1,-1.75) node[ground]{}
(-1,1.5) to[R,i=$i_1$,l^=$R_1$] (-3,1.5)--(-3.5,1.5) node[ground]{}
;
\end{tikzpicture}
\end{center}
\medbreak
On a un montage amplificateur non inverseur idéal. Donc $V_+=V_-=V_e$. Et comme $i_1=i_2$ on a $\frac{V_s}{V_e}=1+\frac{R_2}{R_1} = G_v$ gain du montage. Si $R_1=R_2$ alors $G_v = 2$

\paragraph{Étude de l'AO non idéal tel que : }$R_e \neq \infty$ et $\mu \neq \infty$.\\
Le montage devient alors :\\
\medbreak
\begin{center}
\begin{tikzpicture}[scale = 1]
  \draw[color=black]
  	(1,1) node[op amp] (opamp) {}
  	(opamp.+) node[left] {}
  	(opamp.-) node[left] {}
  	(opamp.out) node[right] {}
  	(opamp.out) -- (3.5,1)
  	(1,-2) node[ground]{} to[V_=$\mu \epsilon$] (1,0)--(1,1)--(opamp.out)
  	(-1,1.5)--(opamp.-)--(-0.5,1.5)--(-0.5,2.5)to[R, l=$R_2$,i<=$i_2$](2,2.5)--(2,1)
  	(-1,-0.2)--(-0.2,-0.2)--(opamp.+)
  	(-1,1.5) to[R, l=$R_e$,i^<=$i_{0d}$] (-1,0)--(-1,-.2)to[V,v^<=$V_e$](-1,-2)node[ground]{}
  	(-1,1.5) to[R,i=$i_1$,l^=$R_1$] (-3,1.5)--(-3.5,1.5) node[ground]{}
  	(3.5,-2)to[open,v_>=$V_s$](3.5,1)
;
\end{tikzpicture}
\end{center}
\medbreak
La loi des noeuds en A donne : $i_1=i_2+i_{ed}$. Ceci nous donne donc :
\begin{align*}
& \frac{V_e-\epsilon}{R_1} = \frac{V_s +\epsilon - V_e}{R_2} + \frac{\epsilon}{R_e}\\
& V_e(\frac{1}{R_1}+\frac{1}{R_2}) =\frac{V_s}{R_2}+\epsilon(\frac{1}{R_e}+\frac{1}{R_1}+\frac{1}{R_2})\\
& \text{or, } \epsilon =\frac{V_s}{\mu}\text{, d'où :}\\
& {G'}_v = \frac{V_s}{V_e} = \frac{\mu.R_e(R_1+R_2)}{R_e(R_1\mu+R_1+R_2) + R_1R_2}
\end{align*}

Si on prend $R_1=R_2=1K\Omega$, $R_e=1M\Omega$ et ${G'}_v=1.9999$.\\
En conclusion, l'influence de $\mu \neq \infty$ et $R_e \neq \infty$ est négligeable.

\paragraph{Influence de la dépendance du gain à la fréquence} $R_e \rightarrow \infty$ et $u(p)=\frac{A_0}{1+\frac{p}{\omega_c}}$ avec $Z_0 = 10^6$ et $\frac{\omega_c}{2\pi}$.\\

On a d'après le calcul précédent et $R_e \rightarrow \infty$:\\
\begin{align*}
{G''}_v &= \frac{\mu(R_1+R_2)}{R_1\mu+R_1+R_2}\\
&=\frac{1}{\frac{R_1}{R_1+R_2}+\frac{1}{\mu}}\\
&= \frac{1}{\beta+\frac{1}{\mu}} \text{\indent avec }\beta=\frac{R_1}{R_1+R_2}=\frac{1}{G_v}\\
&=\frac{1}{\beta(1+\frac{1}{\mu\beta})}
\end{align*}

\noindent L'erreur relative est $\delta = \frac{v^{ideal}_s-v_s}{v_s}$.\\
Or, $\frac{v_e}{\beta}=G_v.v_e=v^{ideal}_s$\\
donc, $v_s(1+\frac{1}{\mu\beta})=v^{ideal}_s$\\
d'où, $\delta=\frac{1}{\mu\beta}$\\

Mais que vaut ${G''}_v$ en fonction de p?\\
\[{G''}_v = \frac{1}{\beta\frac{\beta\mu}{\beta\mu+1}} = \frac{\frac{A_0}{1+\frac{p}{\omega_0}}}{\beta\frac{A_0}{1+\frac{p}{\omega_0}}+1} = \frac{A_0}{\beta A_0+1+\frac{p}{\omega_0}}\]
or, $\beta A_0 >> 1$ donc 
\[{G''}_v \approx \frac{A_0}{A_0\beta+\frac{p}{\omega_0}}=\frac{1}{\beta}\frac{1}{1+\frac{p}{A_0\beta\omega_0}} \]

Le gain statique du montage quand $\omega \rightarrow 0$ est $\frac{1}{\beta}=1+\frac{R_2}{R_1}$.

La fréquence de coupure a -3db du montage est $A_0\beta f_0$.

Le produit gain bande est $\frac{1}{\beta}*A_0\beta f_0$, donc plus le gain est élevé et plus la bande passante est faible.
\bigbreak
\paragraph{Impédance de sortie non nulle}
On considère le montage suivant avec l'impédance de sortie $R_0$ non nulle :\\
\medbreak
\begin{center}
\begin{tikzpicture}[scale = 1]
  \draw[color=black]
(1,1) node[op amp] (opamp) {}
(opamp.+) node[left] {}
(opamp.-) node[left] {}
(opamp.out) node[right] {}

(opamp.out) to  [R, l =$R_0$] (3.5,1)
(1,-2) node[ground]{} to[V_=$\mu \epsilon$] (1,0)--(1,1)--(opamp.out)
(-1,1.5)--(opamp.-)--(-0.5,1.5)--(-0.5,2.5)to[R, l=$R_2$,i<=$i_2$](3.5,2.5)--(3.5,1)
(-1,-0.2)--(-0.2,-0.2)--(opamp.+)
(-1,1.5) to[R, l=$R_e$,i^<=$i_{0d}$] (-1,0)--(-1,-.2)to[V,v^<=$V_e$](-1,-2) node[ground]{}
(-1,1.5) to[R,i=$i_1$,l^=$R_1$] (-3,1.5)--(-3.5,1.5) node[ground]{}
(3.5,-2) node[ground]{} to[R,l=$R_L$,v_>=$V_s$](3.5,1)
(2,-2)to[open,v_>=$V_s'$](2,1)
;
\end{tikzpicture}
\end{center}
\medbreak
Expressions de ${G'''}_v = \frac{V_s}{V_e}$\\
On applique la loi des nœuds en B

\begin{align*}
& \frac{v_A-v_S}{R_2}+\frac{{v'}_s-v_S}{R_0} = \frac{v_S}{R_L}\\
& \frac{v_A-v_S}{R_2}+\frac{{v'}_s-v_S}{R_0} = v_P(\frac{1}{R_L}+\frac{1}{R_2}+\frac{1}{R_0})\\
\end{align*}
or, ${v'}_p=\mu\epsilon$, $v_A=v_e-\epsilon$, et $v_A=\frac{R_1}{R_1+R_2}v_S$\\

\begin{align*}
& {v'}_p=\mu(v_e-\frac{R_1}{R_1+R_2}V_S)\\
& \frac{R_1}{R_2(R_1+R_2)}V_S +\frac{\mu}{R_0}(v_e-\frac{R_1}{R_1+R_2})=v_S(\frac{1}{R_L}+\frac{1}{R_2}+\frac{1}{R_0})\\
& \frac{\mu}{R_0}v_e = v_S(\frac{1}{R_L}+\frac{1}{R_2}+\frac{1}{R_0} + \frac{\mu R_1R_2-R_1R_0}{R_0R_2(R_1+R_2)}) \text{\indent or }uR_2>>R_0\\
& {G'''}_v = \frac{V_s}{V_e} = \mu \frac{1}{\frac{R_0}{R_2}+\frac{R_0}{R_L}+\mu\beta}
\end{align*}
Remarque : si $R_0$ = 0 on retrouve $G_v=\frac{1}{\beta}$.

Pour minimiser l'influence de $R_0$ sur le gain, il faut minimiser $\frac{R_0}{R_2}+\frac{R_0}{R_l}$ donc avoir $R_2 >> R_0$ et $R_L >> R_0$.


\end{document}