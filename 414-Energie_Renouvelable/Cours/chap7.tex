\documentclass[main.tex]{subfiles}
\begin{document}

\emph{Cette partie du cours est construit collaborativement avec M. Hoang et les élèves de l'UE 414.}


\section*{Construction du plan du cours}
On s'interesse d'abord au réseau électrique
\begin{itemize}
\item Smart grid
\item Gestion du  réseau
  \begin{itemize}
  \item pertes
  \item Transfert P Q
  \item Équilibrage (phase, tension, courant)
  \end{itemize}
\item (adaptation des lois de Kirchhoff)
\item DC vs (et ?) AC
\end{itemize}
Lien de recherches:
\begin{itemize}
\item \url{http://server.idemdito.org/electro/elec/gen/couplage.htm}
\item \url{https://fr.wikipedia.org/wiki/Transform\%C3\%A9e_de_Park}
\item \url{}
\end{itemize}
\paragraph{Objectif du cours :} Pouvoir comprendre et expliquer ce qui constitue le schéma distribué.

\section{Cours 1 - Introduction}
\emph{Beaucoup de questions et de réponses pour cerner les besoins et interrogations de la classe}

\paragraph{La terre  électrique} 


\begin{tikzpicture}
  \draw (0,0) rectangle (4,4) (2,2) node{Carcasse};
  \draw (1,0)node[below left]{PE}--  ++(0,-1) to[R]++(-2,0) node[ground]{};
  \draw (3,0) to[open,v<=$v$] ++(0,-2) node[ground]{};
  \draw (5,-1) node[ground]{} to[R,l_=$1k\Omega$] ++(0,3) -- ++(0,2)coordinate(A)-- ++(0,1)++(0,0.5) circle(0.5);
  \draw (A) -- (4,3);
\end{tikzpicture}

{\centering\Huge 30 mA : courant mortel pour l'homme}

\emph{E. H est très chagriné que nous quittions le département EEA avec nos faibles connaissances en conversion d'énergie que nous constatons aujourd'hui}

\paragraph{De la création de l'alternatif}
Tesla, Westing House, Edison,
L'alternatif à pris le dessus, notamment grace à l'industrialisation de la fabrication des transformateurs, qui permettent de convertir (abaisser ou élever) tensions et  courants de manière simple et reproductible. L'alternatif a également un interet pour l'ouverture/fermeture des interrupteurs.

\paragraph{De la création du neutre}
dessin a rajouter.
La création du neutre a lieu dans les transformateurs d'adaption du réseau:
\begin{itemize}
\item primaire : triphasé ``triangle'' .
\item secondaire : triphasé étoile avec un 4ème fil qui ressort : le neutre.
\end{itemize}
\paragraph{Des voitures éléctriques}
C'est le futur. Beaucoup d'application de l'EEA (traitement du signal prédiction ).
Technologies à suivres: Vehicule2Grid Vehicule2Home.
\paragraph{De l'importance des TPs} Les TP ont pour but de vérifier expérimentalement les notions théoriques vues en cours, les comparer, chercher à étudier la différence entre modèle théoriques, simulation numériques et relevé expérimentaux.
+ suite du cours 2


\section{Cours 2}

\paragraph{Problématique}
Quelles sont les propriétés du courant au abords des maisons ? Comment modéliser le réseau alternatif basse tension (réseau domestique) ?


Il n'ya pas ou peu de capteur de courant sur le ``réseau''. 
Le reseau : 220 000 transformateurs ``domestiques'' , on alimente les foyers francais en boucle ouverte. Il existe aussi 4000 poste sources
(très huate tension vers moyenne haute tension)
Sur le réseau l'enjeu est de connaitre en temps réel l'état du système pour adapter la gestion de puissance .

\begin{figure}[H]
  \centering
  \begin{tikzpicture}
    \draw (0,0) to[short,o-] ++(-1,0) to[V,v=$v(t)$ ] ++(0,2) to[short,-o] ++(1,0);
  \end{tikzpicture}
  \[
    V \sqrt{2}\sin(\omega t)
  \]
  \caption{Réseau alternatif BF}
\end{figure}
\subsection{Convertisseur DC- AC}
Comment transfere l'énergie d'une batterie au réseau ?
\emph{cf cours de 411 (NB: le poly de 411 n'est pas encore fini)}

Un convertisseur d'électronique de puissance ne permet que de relier des sources de natures différentes (cf règle d'associations des sources)\\


schéma onduleur. schéma bloc asservissement\\


Structure de commande : Hysterisis (en TP) ou autre

\subsection{Convertisseur DC-DC}
\emph{cf cours de 233  pour plus d'équations}

Application: convertisseur panneau photovoltaïque / bus DC


Il existe deux grandes familles de structures selon la nature "électrique"  des sources et des charges.
\begin{itemize}
	

\item[$\bullet$] \textbf{Structure à transfert direct d'énergie}\\

	On associe deux "sources" de nature différentes (source de tension et de courant ou l'inverse).
	
% \begin{figure}[H]
% 	\centering
% 	\includegraphics[width = 0.6 \linewidth]{transfertDirect}
% 	\caption{Convertisseur statique à transfert d'énergie direct}
% \end{figure}
	
\item[$\bullet$]  \textbf{Structure à transfert indirect d'énergie}\\

	On associe deux "sources" de même natures. L'énergie de la première "source" est stockée dans un élément de stockage puis destockée dans la deuxième "source".
	
	% \begin{figure}[H]
	% 	\centering
	% 	\includegraphics[width = 0.6 \linewidth]{transfertInDirect}
	% 	\caption{Convertisseur statique à transfert d'énergie indirect}
	% \end{figure}
	
\end{itemize}
\subsection{Synthèse de convertisseur}
\begin{center}
	\begin{tabular}{|c|p{7cm}|p{7cm}|}
		\hline
		E/S & courant & tension \\
		\hline
      tension & \textbf{hacheur série} (buck)


		 \begin{circuitikz}[scale=1]
			\draw (0,0) to[V, v=$V_e$,i=$i_v$] (0,2);
			\draw (0,2)to [switch] (1,2) to[L] (3,2)(3,0)--(0,0);
			\draw(3,0) --(4,0) (4,2) to [battery, v<=$V_{DC}$, i=$i_s$] (4,0)(4,2)-- (3,2) ;
			\draw (1,2) to [switch] (1,0);
		  	\end{circuitikz}
	  	\begin{center}
	  	$V_{DC} \leq V_e$
	  	\end{center}
             & \textbf{Hacheur à stockage inductif} (Buck-boost)


		 \begin{circuitikz}[scale=1]
			   	\draw (0,0) to[V, v=$V_e$,i=$i_v$] (0,2);
			   	\draw (0,2)to [switch] (1,2) to[switch] (3,2)(3,0)--(0,0);
			   	\draw(3,0) --(4,0) (4,2)to [battery, v<=$V_{DC}$, i=$i_s$] (4,0) (4,2) -- (3,2) ;
			   	\draw (1,2) to [L] (1,0);
			   \end{circuitikz}
		   \[ |V_{DC}| < V_e \text{ ou } > V_e \]
				   \\
		\hline
      courant
      & \textbf{Hacheur à stockage capacitif} (Cuck)


		 \begin{circuitikz}[scale=1]
			\draw (0,0) to[V, v=$V_e$,i=$i_v$] (0,2);
			\draw (0,2)to [L] (1,2) to[C] (2,2) to [L] (3,2);
			\draw (2,2) to [switch] (2,0);
			\draw (3,0)--(0,0);
			\draw(3,0) --(4,0) (4,2)to [battery, v<=$V_{DC}$, i=$i_s$] (4,0)(4,2)-- (3,2) ;
			\draw (1,2) to [switch] (1,0);
		\end{circuitikz}
	\[ |V_{DC}| < V_e \text{ ou } > V_e \]
	  & \textbf{hacheur parallèle} (boost )


        \begin{circuitikz}[scale=1]
	  	\draw (0,0) to[V, v=$V_e$,i=$i_v$] (0,2);
	  	\draw (0,2)to [L] (1,2) to[switch] (3,2)(3,0)--(0,0);
	  	\draw(3,0) --(4,0) (4,2)to [battery, v<=$V_{DC}$, i=$i_s$](4,0) (4,2)-- (3,2) ;
	  	\draw (1,2) to [switch] (1,0);
	  \end{circuitikz}
  \[ V_{DC} \geq V_e \]\\
		\hline
	\end{tabular}
\end{center}
\begin{rem}

	Si la cellule de  commutation est non réversible en courant ( association transistor+diode), il existe deux mode de fonctionnement (mode de conduction):
	\begin{itemize}
		\item[$\bullet$] Conduction continue (CC)
		\item[$\bullet$] Conduction discontinue (CD)
	\end{itemize}
  \end{rem}
    \begin{rem}
      Dans notre étude la charge est remplacée par une
      \emph{batterie}, connecté à  un bus DC, ainsi la tension
      $V_{DC}$ est imposée ! $R \simeq 0$.

      
    \end{rem}

\subsubsection{Hacheur (buck)}
    \begin{prop}
      Dans le cas d'un hacheur buck on a :
      \[
        <V_e> = \frac{V_{DC}}{\alpha}
      \]
    \end{prop}
\subsubsection{Hacheur (boost)}
\paragraph{M. Hoang} sort du cours a 9h10 pour 10min pour que nous decidions ce que l'on doit faire.  9h33 il n'est toujours pas la



 \begin{prop}
      Dans le cas d'un hacheur boost on a :
      \[
        <V_e> = (1-\alpha) V_{DC}
      \]
    \end{prop}

    \begin{figure}[H]
      \centering
      \begin{circuitikz}
        \begin{axis}
          [axis lines = left,
          xmin = 0,xmax = 5,ymin = 0,ymax =5,ticks=none,ylabel=$I$,xlabel=$v$]
          \addplot[no marks,smooth]{2-exp((x-4)*3)};
          \addplot[no marks,smooth]{3-exp((x-4)*3)};
          \addplot[no marks,smooth]{4-exp((x-4)*3)};
          \addplot[no marks,dashed]plot coordinates {(3,0) (3,5)};
        \end{axis}
      \end{circuitikz}
      \caption{Point de fonctionnement}
    \end{figure}
\section{Cours 4 }





\section{Ressources intéressantes:}
\begin{itemize}
\item \url{http://www.epsic.ch/cours/electronique/techn99/elnthcircuit/cidectxt.html}
\item Podcast France culture mercredi 20/03 : ``la méthode scientifique''
\item Équilibrage du réseau:\\
\url{http://www8.umoncton.ca/umcm-cormier_gabriel/Electrotechnique/Chap4.pdf}
\item Des éclairs, Jean Echenoz
\item Les énergies renouvelables pour la production d'électricité, Leon Freris,Dabid Infield, Dunod
\end{itemize}

\end{document}

%%% Local Variables:
%%% mode: latex
%%% TeX-master: "main"
%%% End:
