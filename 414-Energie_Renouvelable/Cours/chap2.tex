\documentclass[main.tex]{subfiles}
\renewcommand{\epsilon}{\varepsilon}
\begin{document}

\emph{Dans ce chapitre on donne un état des différentes productions d'énergie électriques, pas de calcul, juste de la recopie de slides}

\section{Les différentes sources d'énergies pour l'électricité}
\subsection{Rappels}

\subsubsection{Unités}
\paragraph{Puissance}
\begin{itemize}
\item \item \SI{1}{ch} = \SI{735}{W}
\end{itemize}
\paragraph{Énergie}

\begin{itemize}
\item 1 tonne équivalente pétrole (tep) = \SI{11628}{kWh} = \SI{41,8}{GJ} = \num{7,33} barils de pétroles
\item \num{1} baril de pétrole = \SI{150}{L} = \SI{5,6}{GJ}=\SI{1580}{kWh}
\item \SI{1}{cal} =\SI{4,18}{J}
\item \SI{1}{eV} = \SI{1,6e-19}{J}
\end{itemize}

\paragraph{Conversion d'énergie}

\subsection{Les énergies non renouvelable}
\begin{defin}
  Une énergie non renouvable est une source d'énergie qui se renouvelle moins vite qu'on ne la consomme et de manière négligeable à l'échelle humaine.
\end{defin}
\subsubsection{Les énergies fossiles}
Environ 80\% de l'énergie consommée dans le monde est d'origine fossile. On distingue:
\begin{itemize}
\item le charbon (24\%)
\item le pétrole (35\%)
\item le gaz naturel(21\%)
\end{itemize}

ces ressources permettent d'assurer la plupart des moyens de transports, le chauffage, et production d'électricité et de chaleur (avec éventuellement cogénération\footnote{récupération de la chaleur pour le réseau urbain d'eau chaude})
\subsubsection{Les énergies nucléaires}
\begin{itemize}
\item Par des réactions de fissions
  6\% énergie primaire consommée dans le monde. 15\% de la production mondiale.
  \begin{rem}
    Une ressource potentiellement illimité avec des technologies de neutrons rapides (ex: super phénix en france)
  \end{rem}
\item Par des réacteur de fusion , pour la fin du 21ème siècle.
\end{itemize}

\subsubsection{Problématique actuelle}
Un double problème :
\begin{enumerate}
\item Raréfaction des ressources et dépendance
  \begin{itemize}
  \item Ressources carbonées fossiles: charbon
  \item nucléaire : Uranium (neutrons lents) 200 ans. neutrons rapide, >5000 ans.
  \item Problème géopolitique de la dépendance, 50\% de l'énergie consommée au sein
  \end{itemize}
\item Des conséquences écologiques à limiter
  \begin{itemize}
  \item Limitation des émissions de gaz à effet de serre (380 ppm en 2008, >550 ppm en 2035) pour limiter le réchauffement climatique anthropique.
  \item Gestion des déchets nucléaires
  \end{itemize}
\end{enumerate}

\subsection{Les énergies renouvelable}
C'est la suite du cours.
\begin{enumerate}
\item Énergie hydraulique

\item Énergie éolienne

  \emph{étudié dans le chapitre \ref{chap:eol}}

\item Énergie solaire
  \begin{enumerate}[label=\alph*)]
  \item Le chauffage solaire
  \item Les centrales électriques solaires (thermodynamique, avec cogénération)
  \item Le photovoltaïque (\emph{étudié dans le chapitre \ref{chap:photov})}
  \end{enumerate}

\item Géothermie

  \begin{enumerate}[label=\alph*)]
  \item La production directe de chaleur (ex aquifère de Cachan)
  \item La production d'électricité
  \item Les pompes à chaleur
  \end{enumerate}

\item{La biomasse}

  \begin{enumerate}[label=\alph*)]
  \item Les bio-carburants
  \item Le biogaz
  \item Le bois
  \end{enumerate}
\end{enumerate}

\section{L'hydroélectricité}
\subsection{Sur terre}

\subsubsection{Les prémices}

\subsubsection{Les grandes centrales hydroéléctriques}
\begin{enumerate}
\item Avantages:
  \begin{itemize}
  \item Le caractère intermittent est controllé
  \item Inondation évité en aval
  \item Irrigation
  \item Tourisme
  \end{itemize}
\item Inconvénients:
  \begin{itemize}
  \item Vallée noyée (village et terre agricoles)
  \item Risque d'inondation catastrophique en aval si rupture
  \item Bloque la remontée des poissons
  \item Se remplit petit à petit
  \item Prive de sédiment le cours d'eau en aval
  \end{itemize}
\end{enumerate}

\subsubsection{La petite hydroélectricité}
\subsection{En mer}

\subsubsection{Les marées: prémices}
\begin{itemize}
\item Usine marémotrice de la rance ouverte en 1966 \SI{240}{MW}, \SI{550}{GWh/an}
\item Sihwa (Corée du Sud, \SI{258}{MW})
\end{itemize}
\subsubsection{Les courants permanents}
\subsubsection{La houle}

\end{document}

%%% Local Variables:
%%% mode: latex
%%% TeX-master: "main"
%%% End:
