\documentclass[main.tex]{subfiles}
\begin{document}
\section{Origine de l'énergie solaire}
L'énergie provient du rayonnement émis par le soleil après avoir
parcouru jusqu'a la Terre distante de \SI{1,496e11}{m}.
\subsection{Constitution du Soleil}
Le soleil (de rayon $R_s$=\SI{1,392e9}{m}) peux se décomposer en trois couches:


\begin{enumerate}
\item \textbf{Le coeur:}

  C'est la qu'a lieu la réaction nucléaire de fusion qui libère (beaucoup)
  d'énergie. Cette zone occupe un quart du rayon solaire, et possèd eune
  température de 15 millions de Kelvin.On estime que \SI{4.26}{tonnes}de matière y sont consommé chaque seconde pour  \SI{383e15}{GW} de puissance.
  C'est un processus autorégulé (le soleil ne va pas s'effondrer ou exploser
  dans les années qui viennent).

\item \textbf{La zone de radiation}:

La zone de radiation entre 0.25 et 0.7 du rayon solaire, très dense (98\% de la masse du soleil). Les atomes d'hygrogène et hélium ionisé émettent des photons absorbés par d'autre ions pas de convection thermique.
\item \textbf{La zone de convection:}
  Échange thermique par convection amenant la chaleur vers m'extérieur,on passe de 2 million à 5800K.
  La photosphère  produit le rayonnnement solaire, épaisse d'environ 400km et de température moyenne 5781K.
\end{enumerate}

\subsection{Rayonnement produit et loi utiles}

Le rayonnement produit par le soleil à les caractéristique d'un corp noir:

\[
  B_\nu(T) = \frac{2h\nu^3}{c^2} \frac{1}{exp(\beta h \nu)-1} \text{ou  encore }
  B_\lambda(T) = \frac{2hc^2}{\lambda^5} \frac{1}{exp(\beta h c/\lambda)-1}
\]

Ces expressions se simplifient en faisant des hypothèses sur les niveaus d'énergies:
\begin{itemize}
\item Loi de Rayleight-Jeans  $h\nu \ll kT $ : $B_\nu^{RJ}(T)=\frac{2\nu^2}{c^2}kT$
\item Loi de Wien $h\nu \gg kT$ : $B_\nu^W (T) = \frac{2h\nu^3}{c^2}exp(-\frac{h\nu}{kT})$
\end{itemize}

La puissance surfacique reçue en fonction de la température est elle d'après la loi de Boltzmann:
\[
  P_s = \sigma T^4
\]
\subsection{Notion d'Air-Masse}

\begin{defin}
On nomme \emph{air-masse} l'épaisseur atmosphérique effectivement traversée normalisé à l'épaisseur traversé jusqu'au niveau de la mer pour un soleil au zénith, en condition normale de pression:
\[
  m = \frac{P}{1013}\frac{1}{sin(\alpha)}exp\left(\frac{-z}{7.8}\right)
\]
\begin{itemize}[label=--]
\item $P$ : Pression atmosphérique en hPa ou millibar
\item $\mathbf{\alpha}$ : élévation du soleil sur l'horizon
\item $z$ :  altitude en km (7.8 km est l'épaisseur moyenne de l'atmosphère)
\end{itemize}
\end{defin}

On défini alors les conditions d'ensoleillement par les lettres AM suivi de $m$:
  \begin{itemize}
  \item AM0 correspond aux conditions hors atmosphère
  \item AM1 au sol lorsque le soleil est au zénith
  \item AM2 au sol lorsque le soleil est à $30^o$ sur l'horizon.
  \end{itemize}

En pratique le flux reçu ne dépasse 1000$W/m^2$ (1367 $W/m^2$ pour AM0).
Les conditions standartd des qualification des cellules sont un spectre $AM1.5$, une puissance incidente de 1000$W/m^2$ et une température de $25^o$C.


\emph{blabla sur le rayonnement direct et indirect}


\section{Principe de la conversion: la cellule photovoltaïque}
\subsection{Historique}
Le developpement de l'exploitation de l'énergie solaire s'est fais au long des découvertes scientifiques majeures du XIXème et XXème siècle:

\begin{description}
\item[1839] Découverte de l'effet photovoltaïque par Antoine Becquerel sur un couple electrochimique
\item[1877] 1ère cellule PV au sélénium
\item[1954] 1ère cellule PV au silicium rendement de 4,5\% à 6\%.
\item[1955] 1ère commercialisation de cellule PV 14mW (2\%).
\item[1958] Satellite Vanguard-1 avec des cellules PV qui fonctionnent pendant 8 ans.
\item[Années 60] Montée des rendements et puissances.
\item[1963] Japon : 242 W sur une maison
\item[1970] Mission Solar One (Université du Delaware)
\item[1981] Premier avion Solaire ``Solar Challenger'' Paris-Manston (Angleterre) 262km.
\end{description}

\subsection{La jonction PN}
\emph{cf UE 232}
\subsection{Effet photovoltaïque}

Un photon suffisament énergétique peux créer une paire électron/trou dans la zone de transition, contribuant ainsi à augmenter le courant inverse (contribution du courant de génération-recombinaison).

Il faut pour cela que l'énergie du photon soit supérieur à l'énergie de gap (Pour le silicium $E_g=1.1eV$).

\subsection{La photodiode}


\begin{figure}[H]
  \centering
  \begin{tikzpicture}
    \draw[fill=gray!10] (0,0) rectangle (7,4);
    \draw[fill=gray!40] (0,1) rectangle ++(7,2);
    \draw (0.5,5) node[above]{K} -- (0.5,4)
    (0.5,-1) node[below]{A} -- (0.5,0)
    (0.5,0.5) node{N} (0.5,2.5) node{I} (0.5,3.5)node{P};
    \draw[latex-latex] (3.5,2.7) node(E)[left]{$e^-$} -- ++(2,-1) node[right](T){$t^+$} coordinate[midway](P);
    \node (Pstart) at (6.5,5.2) {$h\nu$};
    \draw[-latex,decorate,decoration={snake,amplitude=.4mm,segment length=2mm,post length=1mm}] (Pstart) -- (P);

    \draw (11,0)node[below]{A} to[photodiode] ++ (0,4) node[above]{K};
  \end{tikzpicture}
  \caption{Schéma de la photodiode}
\end{figure}


Pour des photon d'énergie $E=h\nu$ on créer un courant:
\[
  I_{ph} = q_e \eta \phi_p = q_e \eta \frac{\lambda}{hc} \phi_e
\]
où
\begin{itemize}
\item $q_e$ charge électrique.
\item $\eta$ rendement quantique.
\item $\phi_p$ fux de photon reçu par la photodiode.
\end{itemize}

\subsubsection{Caractéristique courant-tension}

\begin{prop}[Rappel sur la diode]
  \begin{itemize}
  \item Une étude théorique (développé dans le cours UE 232 de Saphire) permet
    de montrer que la caractéristique de la diode en convention générateur:

\[
  I_d = - I_s \left(exp\left(\frac{q_eV}{kT}\right)-1\right)
\]

On a typiquement $I_s = $\SI{10e-12}{A.cm^{-2}} et$k = k_B = $\SI{1.3807e-23}{J/K}.


\item  En ajoutant le courant issus de la création de la paire électron trou on a le courant d'une photodiode:
  \[
    I = I_{ph}- I_s \left(exp\left(\frac{q_eV}{kT}\right)-1\right)
  \]
\end{itemize}
\end{prop}

\subsubsection{Schéma équivalent d'une photodiode}


On peux construire le schéma équivalent:
\begin{figure}[H]
  \centering
  \begin{tikzpicture}
    \draw
    (0,0) to[I,i=$I_{ph}$] ++(0,2) -- ++(1,0) coordinate(A) to[diode,i>^=$I_d$] ++(0,-2) -- (0,0)
    (A) -- ++(1,0) to[R,l=$R_s$,i=$I_p$] ++(3,0) to[open,v<=$V_p$]++(0,-2) --++(-3,0)
    (A) ++(1,0) to[R,l=$R_{sh}$] ++(0,-2) -- (0,0);
  \end{tikzpicture}
  \caption{Schéma équivalent}
\end{figure}
Avec:
\begin{itemize}
\item Resistance de contact et connexion: $R_s=5$ à $20m\Omega$
\item Fuite de courant aux bords de la jonction : $R_{sh}=20$ à $200\Omega$/
\end{itemize}

\subsection{Devenir des photons}


quand un photon frappe un matériaux de silicium il peut se  produire trois phénomènes:
\begin{itemize}
\item Le photon peut traverser le silicium (photon à faible énergie)
\item Le photon peut se réfléchir sur la surface(30\% silicium, descend à 3\% avec une couche anti reflet)
\item Le photon peut être absorbé par le matériau. Si l'énergie du photon est supérieur à l'énergie de gap celui ci génère une paire électron trou et un échauffement.
\end{itemize}
\begin{rem}
  Les photons UV ne génère qu'une seule paire d'electron-trou, le reste de l'énergie est convertie en chaleur. Ce qui représente environ 28\% de perte  de l'énergie incidente.
\end{rem}

\subsection{Rendement}

La répartition spectrale de la puissance émise par le soleil est telle qu'elle est constituée majoritairement de photon possédant une énergie supérieure à la bande interdite du silicium (1.1eV). L'énergie en excès est convertie en chaleur (photons: AM1.5 25\% en théorie).

Le rendement maximal théorique est de 50\% à AM1.5.

\begin{prop}
  En pratique 25\% de l'énergie du rayonnement solaire est convertie en electricité utile.
\end{prop}


\subsection{Caractéristique du générateur}
\begin{defin}
  La cellule photovoltaïque est un générateur élementaire à courant continu qui convertit directement l'énergie lumineuse en énergie électrique.

\end{defin}
La tension est basse car inférieur à la tension de polarisation directe de la jonction (0.5 V à 0.8V pour le silicium)

\begin{prop}
  La puissance crète pour une cellule 10cm x 10cm est de l'ordre du Watt: 0.6V , 5A soit 3W pour du silicium monocristallin.
\end{prop}

\section{Mise en oeuvre}


On assemble différentes élements photovoltaique en série et parallèle pour augmenter la tension et (resp.) le courant disponible.

Les cellules solaire utilisent des matériaux sous différentes formes, avec de différents rendements :


\subsection{CAractéristique d'une cellule}






\end{document}

%%% Local Variables:
%%% mode: latex
%%% TeX-master: "main"
%%% End:
