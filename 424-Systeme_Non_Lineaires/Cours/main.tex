\documentclass[openany]{../../cours}
\usepackage{../../raccourcis}
% Mise en page
\author{Pierre-Antoine Comby}
\teacher{Mohamed Abbas Turkis}
\module{UE424}
\title{Commandes de système non linéaires}
\usepackage{multicol}
\usetikzlibrary{decorations.markings}
\usetikzlibrary{patterns}

\let\epsilon\varepsilon
\begin{document}

\maketitle
\hfill
\begin{center}
  \textsc{Avertissement:}\\
  La structure de ce cours est purement fictionnelle. Toute ressemblance avec un plan existant ou ayant existé serait tout à fait fortuite.
\end{center}
\hfill
\newpage

\tableofcontents
\part{Études des systèmes non linéaires}
\chapter{Classification}
\subfile{chap1.tex}
\chapter{Caractérisation de la stabilité}
\subfile{chap2.tex}
\chapter{Linéarisation}
\subfile{chap3.tex}
\chapter{Methode du premier harmonique}
\subfile{chap4.tex}
\chapter{Stabilité des systèmes non linéaires}
\subfile{chap5.tex}
\part{Commande des systèmes non linéaires}
\chapter{Commandabilité et observabilité en non linéaire}
\subfile{chap6.tex}
\chapter{Commande par bouclage linéarisant}
\subfile{chap7.tex}
\chapter{Commande hiérarchisée}
\subfile{chap8.tex}
\chapter{Rejet de pertubation et commande Robuste}
\subfile{chap9.tex}

\end{document}

%%% Local Variables:
%%% mode: latex
%%% TeX-master: t
%%% End:
