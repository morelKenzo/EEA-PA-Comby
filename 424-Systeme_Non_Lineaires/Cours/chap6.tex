\documentclass[main.tex]{subfiles}
% Relu jusqu'à 3.4 inclus, X 08/02/2015
% Corrigé jusqu'au 4.3 inclus, A 28/02/2015.
\newcommand{\Lc}{\mathcal{L}}
\newcommand{\D}{\mathcal{D}}
\begin{document}
\section{Introduction (notations maths)}

\begin{defin}
  On appelle \emph{champ de vecteur} toute  application de $\R^n \rightarrow \R^n$.
\end{defin}

\begin{defin}
Soit $f : \R^n \rightarrow \R^n$ et $g : \R^n \rightarrow \R^n$, on définit le \emph{crochet de Lie} :
\[ [f,g] :
  \begin{cases}
\R^n & \rightarrow \R^n \\ x & \mapsto J_g(x)f(x) - J_f(x)g(x)
\end{cases}
\]
où $J_f$ et $J_g$ sont respectivement les matrices jacobiennes de $f$ et $g$.
\end{defin}

\begin{prop}[Crochet de Lie]
Soient $f, g \text{ et }h$ des champs de vecteurs et $\lambda_1, \lambda_2 \in \K, (\K = \R \text{ ou } \C)$.
Alors
\begin{align*}
[\lambda_1 f + \lambda_2 g, h ] =  \lambda_1[f,h] + \lambda_2[g,h] \quad & \text{Bilinéaire} \\
[f,g] = - [g,f] \quad & \text{Anti-symétrique} \\
[f,[g,h]] + [h,[f,g]] + [g,[h,f]] = 0 \quad & \text{Identité de Jacobi} \\
[f,f] = 0 \quad
\end{align*}
\end{prop}
\newpage
\begin{defin}
$G$ est une \emph{algèbre de Lie} sur $\K$ si $G$ est un espace vectoriel ayant pour loi interne le crochet de Lie.
\end{defin}

\begin{rem}
Cette définition se restreint au cas qui nous intéresse ici, ce n'est pas la définition générale.
\end{rem}

\begin{rem}
$\Lc(E)$ est l'algèbre de Lie ayant pour famille génératrice l'ensemble des champs de vecteurs $E$.
\end{rem}

\underline{Notation} : Crochet de Lie itéré

$ad_f^0 (x) = g(x)$

$ad_f^1 g(x) = [f,g](x)$

$ad_f^k g(x) = [f,ad_f^{k-1}g](x)$

\begin{defin}
la \emph{dérivée de Lie} d'une fonction $\alpha : \R^n \rightarrow \R$ dans la direction de $f : \R^n \rightarrow \R^n$, notée $L_f\alpha$, est définie par :
\[L_f \alpha(x) = \sum_{i=1}^n \derivp[\alpha(x)]{x_i}f_i(x) \]

Ainsi,
\[L_f^k \alpha (x) = J_{L_f^{k-1} \alpha} (x) f(x) = [ \derivp[L_f^{k-1} \alpha(x)]{x_1} \dots\derivp[L_f^{k-1} \alpha(x)]{x_n}] \vect{f_1(x) \\ \vdots \\ f_n(x) } \]
\end{defin}

\begin{rem}
\begin{itemize}
\item $L_f^0(\alpha(x)) = \alpha(x)$
\item Soient 2 champs de vecteurs $f$ et $g$, alors
\begin{align*}
L_g L_f \alpha (x) & = J_{L_f \alpha}(x) g(x) \\
L_{[f,g]} \alpha(x) & = L_f L_g \alpha(x) - L_gL_f \alpha(x)
\end{align*}
\end{itemize}
\end{rem}

\begin{defin}
La \emph{dimension} d'un ensemble de champs de vecteurs $E=\{f_1(x) \dots f_n(x)\}$, où $f_i(x) : \R^n \rightarrow \R^n$, est la dimension de l'espace vectoriel $\Delta(x)$ engendré par l'ensemble $E$.

\begin{rem}
On fait la confusion entre rang et dimension.
\end{rem}
\end{defin}

\begin{exemple}
\[ f_1(x) = \vect{x_1 \\ x_2 \\ 2}, f_2(x) =
  \begin{bmatrix}
x_1 & x_3 \\ x_2 & x_3 \\2 & x_3
\end{bmatrix}
\text{ et }f_3(x) = \vect{x_2 \\ x_2 \\ 0} \]

Si $x_2 = 0$, alors $\Delta(x) = vect\{( \vect{x_1 \\ 0 \\ 2} ) \} \text{ et }dim=1$.

Si $x_2 \neq 0$, alors $\Delta(x) = vect\{(\vect{x_1 \\ x_2 \\ 2},\vect{1 \\ 1 \\ 0})\} \text{ et }dim=2$.
\end{exemple}

\section{Commandabilité (atteignabilité, contrôlabilité)}

Soit le système non-linéaire (1) (affine en la commande)
\[ \dot{x} = f(x) + g(x)u = f(x) + \sum_{i=1}^m g_i(x) u_i, \quad x \in \R^n \text{ et }u \in \R^m \]

\begin{defin}
Un système est\emph{ commandable} ssi $\forall x \in \R^n, \exists u$ tel que $x$ est atteignable dans un temps fini.
\end{defin}

\begin{thm}[Théorème de Commandabilité]
Le système (1) est commandable ssi la sous-algèbre de Lie $\D = \{g_1 \dots g_m, \Lc(E)\}$ avec $E=\{g_1 \dots g_m,f\}$ est de dimension $n$.
\end{thm}

\begin{exemple}[cas linéaire]
\[ \dot{x} = Ax + Bu \]

\[ E = \{Ax,B\}, [B,Ax] = AB \]
\[ [AB,Ax] = A^2B, \dots, A^{n-1}B, \dots \]
\[ \Lc(E) = vect \{AB,A^2B,\dots\} \]

suivant Cayley Hamilton:
\[ \D = \{B,vect \{AB,AB^2,\dots,A^{n-1}B\}\}\]

$dim \D = rang (B AB \dots A^{n-1}B)$ théorème de Kalman
\end{exemple}

\section{Observabilité (distingabilité)}
Soit le système NL (2) (affine en la commande) :
\begin{align*}
\dot{x} & = f(x) + g(x)u \\
y & = h(x)
\end{align*}

\begin{defin}
Un système est \emph{observable} si $\forall x_1,x_2 \in \R^n$ 2 conditions initiales telles que $x_1 \neq x_2$, $\exists$ une commande $u$ admissible telle que les sorties soient distinctes, $\forall t \geq t_0$ ($t_0$ instant initial).
\end{defin}

\begin{defin}
$\mathcal{V}$ est \emph{l'espace d'observabilité} constitué de toutes les combinaisons linéaires obtenues à partir des dérivées de Lie $L_f$ et $L_g$ des fonctions $h_j(x),j=1 \dots p$ telles que $y\in\R^p$
\[ \mathcal{V} = \{h_j,L_fh_j, L_g h_j, L^2_f h_j,\dots L_g L_f h_j, L_f L_g h_j,\dots \}\]

On note $\nabla \mathcal{V}$ l'ensemble des différentielles (gradient) des éléments de $\mathcal{V}$ :

\[ \nabla \mathcal{V} = \{ \nabla h_j, \nabla L_f h_j ... \} \]
\end{defin}

\begin{thm}[Théorème d'observabilité]
Le système (2) est localement observable en $x_0$ si $\dim \nabla \mathcal{V}(x_0) = n$ et il est observable si $\forall x \in \R^n, \dim \nabla \mathcal{V}(c) = n$
\end{thm}

\begin{exemple}[Cas linéaire]
\begin{align*}
\dot{x} & = Ax + Bu = f(x) + g(x)u\\
y & = Cx = h(x)
\end{align*}

\begin{align*} 
  \mathcal{O}      =  &\{ h(x), L_fh(x), L_gh,
                       L^2_fh ,L_g^2h , L_fL_gh ,
                       L_gL_fh \dots \}                        \\
\mathcal{O}         = &\{ Cx, C.Ax (=L_fh(x)),                 \\
                      & C.B (=L_gh), CA^2x (=L^2_fh),             \\
                      & 0 (=L_g^2h) , 0 (=L_fL_gh) ,              \\
                      & CAB (=L_gL_fh) \dots \}                   \\
  \nabla \mathcal{O}     = &\{ C , CA , 0 ,CA^2 , 0 , 0 , 0 \dots \} \\
\end{align*}
\[
\dim \nabla \mathcal{O}  = {\rm rg} \vect{ C \\ CA \\ CA^2 \\ \vdots \\ CA^{n-1}} \quad \text{Critère de Kalman}
\]
\end{exemple}

\begin{rem}
l'action de la commande intervient dans l'observabilité. Cette contrainte est écartée dans le cas linéaire.
\end{rem}

\end{document}

%%% Local Variables:
%%% mode: latex
%%% TeX-master: "main"
%%% End:
