\documentclass{../../td}
\usepackage{../../raccourcis}
\begin{document}
\subsection*{Exercice 1: Stabilité en non linéaire}
\begin{enumerate}
\item

LE bout de l'autre génie.


\[x(t) = x_0 \frac{1+t_0}{1+t}\]
\[\forall  \epsilon > 0 \exists \delta > 0 \text{ tq } ||x_0|| \leq \delta \Rightarrow ||x(t) || \leq \epsilon\]
contraposée
\[\exists \epsilon \text{ tq } \forall \delta >0, ||x_0|| \leq \delta \et ||x(t) || \geq \epsilon\]
\begin{align*}
|x(t)| = |x_0| \frac{1+t_0}{1+t} &> \frac{|x_0|t_0}{1+t}\\
& > |x_0| = \epsilon \text{ si } t_0 \rightarrow \infty
\end{align*}

\end{enumerate}


\subsection*{Exercice 2: Pendule simple}

\begin{enumerate}
\item
\begin{enumerate}
\item On applique le PFD selon l'axe $u_\theta$ :\\
\begin{align*}
m\ddot{\theta}l &= -mg.sin(\theta)\\
\ddot{\theta} &= -\frac{g}{l} sin(\theta)
\end{align*}

\item On pose le vecteur $x = \begin{pmatrix}x_1\\x_2\end{pmatrix} = \begin{pmatrix}\theta \\ \dot{\theta}\end{pmatrix}$, on a donc le système d'état:\\
\[ \dot{x} = \begin{pmatrix}\dot{x_1} \\ \dot{x_2}\end{pmatrix}\ = \begin{pmatrix}x_2 \\ -\frac{g}{l} sin(x_1)\end{pmatrix}\]\\

\item On a $E = \frac{1}{2}m v^2$ avec  $v= l\dot{\theta}$ :
\[\boxed{E = \frac{ml^2}{2} x_2^2}\]
Et $P = mgl(1-cos(\theta))$:
\[\boxed{P = mgl(1-cos(x_1))}\]

\item On pose $V(x) = E + P$ d'où en zéro:
\begin{align*}
V(0) &= \frac{ml^2}{2} x_{20}^2 + mgl(1-cos(x_{10})) = 0
\intertext{si $x \neq 0$ et $|x_1| < 2 \pi$, alors $cos(x_1)<1$ donc $E+P >0$}
\frac{\partial V}{\partial x_1} &= mgl.sin(x_1)\\
\frac{\partial V}{\partial x_2} &= ml^2 x_2\\
\dot{V}(x) &= \dot{x_1} mglsin(x_1) + ml^2x_2\dot{x_2}\\
\dot{V}(x) &= \dot{x_1} mglsin(x_1) + ml^2x_2(-\frac{g}{l}sin(x_1)) = 0\\
\end{align*}
Pour $|x_1| < \pi$, l'origine sera stable

Il n'existe pas de Q(x) tel que $\dot{V(x) \leq -Q(x)}$\\
Barhashin : $\dot{V}(x) = \{ x_2 \in \R \et |x_1| < 2\pi \}$ \\
Pas de stabilité asymptotique

\end{enumerate}
\item
\begin{enumerate}

\item On applique le PFD selon l'axe $u_\theta$ :\\
\begin{align*}
m\ddot{\theta}l &= -mg.sin(\theta) - \alpha l \dot{\theta}\\
\ddot{\theta} &= -\frac{g}{l} sin(\theta) -\alpha \frac{1}{m}\dot{\theta}
\end{align*}

\item On pose le vecteur $x = \begin{pmatrix}x_1\\x_2\end{pmatrix} = \begin{pmatrix}\theta \\ \dot{\theta}\end{pmatrix}$, on a donc le système d'état:\\
\[ \dot{x} = \begin{pmatrix}\dot{x_1} \\ \dot{x_2}\end{pmatrix}\ = \begin{pmatrix}x_2 \\ -\frac{g}{l} sin(x_1)- \frac{\alpha}{m}x_2\end{pmatrix}\]\\

\item On a $E = \frac{1}{2}m v^2$ avec  $v= l\dot{\theta}$ :
\[\boxed{E = \frac{ml^2}{2} x_2^2}\]
Et $P = mgl(1-cos(\theta))$:
\[\boxed{P = mgl(1-cos(x_1))}\]

\item On pose $V(x) = E + P$ d'où en zéro:
\begin{align*}
V(0) &= \frac{ml^2}{2} x_{20}^2 + mgl(1-cos(x_{10})) = 0
\intertext{si $x \neq 0$ et $|x_1| < 2 \pi$, alors $cos(x_1)<1$ donc $E+P >0$}
\frac{\partial V}{\partial x_1} &= mgl.sin(x_1)\\
\frac{\partial V}{\partial x_2} &= ml^2 x_2\\
\dot{V}(x) &= \dot{x_1} mglsin(x_1) + ml^2x_2\dot{x_2}\\
\dot{V}(x) &= \dot{x_1} mglsin(x_1) + ml^2x_2(-\frac{g}{l}sin(x_1))-\frac{\alpha}{m}x_2 - l^2\alpha x_2^2 \leq 0\\
\intertext{Pour $|x_1| < \pi$, l'origine sera stable}
\end{align*}
Il n'existe pas de Q(x) tel que $\dot{V(x) \leq -Q(x)}$\\
Barhashin : 
\begin{align*}
\dot{V(x)} = 0 \Rightarrow x_2 =0 \Rightarrow \dot{x_1} = 0\\
\text{ pour } |x_1- < \pi \Rightarrow x_2 = 0\\
\text{ donc } \dot{x_2} = 0 \Rightarrow sin(x_1) = 0
\end{align*}
Si $x_1 = 0$, $\pi$ , $-\pi$ il n'y a pas de stabilité asymptotique.\\

L'origine est stable asymptotiquement pour $(x_1,x_2) \ in ]-\pi ; \pi[\times \R$
\[
V(x) = \begin{pmatrix}
x_1 & x_2\end{pmatrix}. \begin{pmatrix}
\frac{\alpha^2}{2m^2} & \frac{\alpha}{2m} \\ \frac{\alpha}{2m} & 1 \end{pmatrix} \begin{pmatrix}
x_1 \\x_2
\end{pmatrix} + \frac{g}{l}(1-cos(x_1))\]
\[V(0) = 0\]
\[V(x) = \begin{pmatrix}
x_1 & x_2\end{pmatrix} P \begin{pmatrix}
x_1 \\ x_2
\end{pmatrix} + \frac{g}{l}(1 - cos(x_1))\]
Or, $P >0$ car:
\[\frac{\alpha^2}{2m^2}>0 \et \frac{\alpha^2}{2m^2} - \frac{\alpha}{4m^2}>0 \et |x_1| < 0\pi\]
donc:

\[P \begin{pmatrix}
P_{11} & P_{12} \\
P_{12}^T & P_{22}
\end{pmatrix} >0 \Leftrightarrow (Lemme de Schur) P_{11} >0 \et (peutetre) P_{11} - P_{12}P_{22}^+P_{12}^T
 >0 (P^+ est la pseudo inverse)
\]
\begin{align*}
\dot{V}(x) &= -\frac{1}{2}(\frac{g\alpha}{lm})x_1 sin(x_1) - \frac{\alpha}{2m} x_2^2  \leq 0\\
&\leq -\frac{g\alpha}{4lm} x_1sin(x_1) - \frac{\alpha}{4m}x_2^2 = Q(x)
\end{align*}

L'origine est localement asymptotiquement stable.
\end{enumerate}


\subsection*{Exercice 3: Exemple de systèmes}

\begin{enumerate}
\item \begin{enumerate}
On pose $V(x) = \frac{1}{2}(x_1^2 + x_2^2)$, et on a bien $V(0) = 0$.
$\dot{V}(x) = x_1\dot{x_1} + x_2 \dot{x_2} = -(x_1^2 + x_2^2)$.\\
On a alors $V(x) \leq -Q(x)$ avec $Q(x) = \frac{x_1^2 + x_2^2}{2}$, donc l'origine est globalement asymptotiquement stable.\\

\item Vérifions la stabilité exponentielle: 
$\exists \alpha>0$, $\beta>0$, $\gamma>0$, et $x>1$ tel que $\dot{V} \leq - \gamma||x--^c$ et $\alpha||x||^c \leq V(x) \leq \beta ||x||^c$\\
Pour la norme euclidienne, on prend $c=2$ et avec $\gamma = 1$\\
$\dot{V} \leq -\gamma||x||^2$\\
Avec $\alpha = \frac{1}{4}$ et $\beta = 1$ la condition 2 est respectée donc on a la stabilité exponentielle.
\end{enumerate}

\item On considère le système suivant:
  \[       \begin{cases}
      \dot{x_1} = x_2 + x_3^4 \\
      \dot{x_2} = -5sin(x_1) - x_2 + u_1 \\
      \dot{x_3} = -kx_3 + u_2 \\
    \end{cases}
\]
\begin{enumerate}
\item Pour $u_1 = u_2 = 0$, l'origine est stable car $\dot{x} = 0 $.\\

\item Pour linéariser, on passe par la matrice du jacobien prise en (0,0,0):
\[
\begin{pmatrix}
\delta \dot{x_1} \\\delta \dot{x_2} \\\delta \dot{x_3}
\end{pmatrix} = \begin{pmatrix}
0 & 1& 0\\-5 & -1 & 0\\0 & 0& -k
\end{pmatrix}.\begin{pmatrix}
\delta x_1 \\ \delta x_2 \\ \delta x_3
\end{pmatrix} = A \begin{pmatrix}
\delta x_1 \\ \delta x_2 \\ \delta x_3
\end{pmatrix}\]
\[
det(\lambda I - A) = (\lambda  + k)(\lambda^2 + \lambda + 5)
\]
La stabilité suivant le critère de Routh donne que c'est stable si $k>0$

\item \[V(x) = 5(1-cos(x_1)) + \frac{1}{2} \dot{x_2}^2 + \frac{3}{2k}x_3^4
|x_1| < 2\pi\]
\[\dot{V}(x) = 5x_3^4sinx_1 - x_2^2 \leq -x_3^4 - x_2^2 \leq 0 \text{(ne marche pas car ne dépend pas de $x_1$)}\]
\[\dot{V}(x) \leq \frac{5}{2} x_3^4 sin x_1 -\frac{1}{2}x_2^2 - 3 x_3^4 \leq -\frac{1}{2}x_3^4 - \frac{1}{2}x_2^2\]
\[Q(x) = -\frac{5}{2}x_3^4sinx_1 + \frac{1}{2}x_2^2 + 3x_3^4 \geq 0 \text{ a condition que } |x_1| < \pi
\]
\end{enumerate}

\end{enumerate}
\end{enumerate}
\end{document}
    

%%% Local Variables:
%%% mode: latex
%%% TeX-master: "../main"
%%% End:
