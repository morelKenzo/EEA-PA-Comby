\documentclass{../../td}{subfiles}
\usepackage{../../raccourcis}
\begin{document}
\subsection*{Exercice}
On considère le système suivant:
\[\left\{\begin{matrix}
\dot{x_1} = x_3^2 + u_1 + u_2\\
\dot{x_2} = x_2x_3 + u_2\\
\dot{x_3} = u_1 \\
y_1 = x_1 - x_3
y_2 = x_2
\end{matrix} \right. \]

\begin{enumerate}
\item Pour vérifier si le système est linéarisable par retour statique on commence par calculer les dérivées successives:
\begin{align*}
\dot{y_1} &= \dot{x_1} - \dot{x_3} = x_3^2 + u_2 \Rightarrow r_1 = 1\\
\dot{y_2} = x_2x_3 + u_2 \Rightarrow r_2 = 1
\intertext{ainsi r= 2}
\begin{pmatrix}\dot{y_1}\\ \dot{y_2}\end{pmatrix} = \begin{pmatrix} x_3^2 \\x_2 x_3 \end{pmatrix} + \begin{pmatrix}0 & 1 \\ 0 & 1\end{pmatrix} \begin{pmatrix} u_1 \\ u_2\end{pmatrix}
\end{align*}
u s'exprime donc en fonction de $D^{-1}$, D n'est pas inversible implique qu'il n'y a pas de bouclage linéarisant statique.\\

\item Pour le bouclage dynamique, les commandes sont dépendantes du temps.
Si on ne rajoute pas de dynamique, on ne trouve pas un r assez grand, on rajoute donc $\dot{x_4} = \omega$ et $\dot{u_2} = \omega$ est une nouvelle commande:
\begin{align*}
\ddot{y_1} &= 2x_3\dot{x_3} + \dot{u_2}\\
&=2 x_3u_1 + \dot{u_2} \text{donc $r_1 = 2$}\\
\ddot{y_2} &= \dot{x_2} x_3 + x_2 \dot{x_3} + \dot{u_2}\\
&= x_2x_3^2 + x_4x_3 + x_2x2u_1+\omega
\end{align*}
 
Ainsi, le système se met sous forme normale:
\begin{align*}
z_1 &= y_1\\
z_2 &= y_2\\
\dot{z_1} &= z_3\\
\dot{z_3} &= 2x_3u_1 + \omega = v_1\\
\dot{z_2} &= z_4\\
\dot{z_4} &= x_2x_3^2 + x_4x_3 + x_2u_1 + \omega = v_2
\intertext{ainsi, on a:}
\begin{pmatrix}
v_1\\v_2
\end{pmatrix} &= \begin{pmatrix}
0\\x_2x_3^2 + x_4x_3
\end{pmatrix} + \begin{pmatrix}
2x_3 & 1\\x_2 & 1
\end{pmatrix}\begin{pmatrix}
u_1 \\ \omega
\end{pmatrix}
\end{align*}

Ainsi, D(x) (la matrice devant le vecteur de commande, hein!) est inversible si $2x_3-x_2 \neq 0 $.\\
Si cette condition est réalisable alors le modèle est linéarisable.

%\imgt{1}


\end{enumerate}


\end{document}
