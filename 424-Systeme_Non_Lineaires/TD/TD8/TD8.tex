\documentclass{../../td}{subfiles}
\usepackage{../../raccourcis}
\begin{document}
\subsection*{Exercice I: Platitude}
\begin{enumerate}
\item On considère le système suivant:
\[
  \begin{cases}
\dot{x_1} &= x_2 - x_1 cos x_1\\
\dot{x_2} &= (x_2^2 + u)(5 + sinx_1)
\end{cases}
\]

Montrons que $x_1$ est une sortie plate, pour cela, il faut exprimer u en fonction de $x_1$ et ses dérivées uniquement:
\begin{align*}
x_2 &= \dot{x_1} + x_1 cos x_1\\
u &= \frac{\dot{x_2}}{5 + sin x_1} - x_2^2\\
&= \frac{\ddot{x_1} + \dot{x_1}cosx_1 - \dot{x_1}x_1 sinx_1}{5 + sin x_1} - (\dot{x_1} + x_1 cos x_1)^2
\end{align*}
$x_1$ est bien une sortie plate.

\item On considère le système suivant:
\[ \left\{\begin{matrix}
\dot{x_1} &= -x_1^2 + x_2\\
\dot{x_2} &= x_2x_1 + u_1\\
\dot{x_3} &= u_2
\end{matrix}\right. \]
Montrons que $x_1$ et $x_3$ sont des sorties plates:
\begin{align*}
x_2 &= \dot{x_1}- x_1^2 \\
u_1 &= \dot{x_2} - x_2 x_1\\
&= \ddot{x_1} - 3x_1 \dot{x_1} + x_1^3\\
u_2 &= \dot{x_3}
\end{align*}
On a bien les commandes en fonctions de $x_1$,$x_3$ et leurs dérivées uniquement.

\end{enumerate}

\subsection*{Exercice II: Planification}
On considère le système suivant:
\[
  \begin{cases}
\dot{x_1} &= x_2\\
\dot{x_2} &= \alpha\dot{x_1} + u
\end{cases}
\]
\begin{enumerate}
\item Trouvons la sortie plate y. On remarque que pour $y=x_1$ on a $u = \ddot{y} - \alpha \dot{y}$, donc ce y convient (est une sortie plate) et on a alors le système:
\[ \left\{\begin{matrix}
\dot{x_1} &= y\\
\dot{x_2} &= \dot{y}\\
u &= \ddot{y} - \alpha \dot{y}
\end{matrix}\right. \]

\item 
\begin{align*}
y_c(t) &= a(T-t)^3 + b(T-t)^2 + c(T-t) + d\\
y_c(0) &= y_0 \Rightarrow aT^3 + bT^2 + cT + d = y_0\\
y_c(T) &= y_T \Rightarrow d = y_T\\
\dot{y_c}(T) &= 0 \Rightarrow -c = 0\\
\ddot{y_c}(T) &= 0 \Rightarrow b = 0\\
&\Rightarrow aT^3 = y_0 - y_T\\
&\Rightarrow a = \frac{y_0 - y_T}{T^3}\\
u &= \ddot{y_c} - \alpha \dot{y_c}\\7
&=3\frac{y_0 -y_T}{T^3}(T-t)(3 + \alpha(T-t)
\end{align*}
On a la commande en BO. En posant $ \delta x = x - x_c$ on peut linéariser le modèle autour du point et approcher une trajectoire point par point.\\
La planification de la trajectoire permet de trouver un modèle linéaire autour de la trajectoire obtenue via $u_c$, la commande de planification.

\end{enumerate}

\subsection*{Exercice III: Suspension magnétique}
On peut appliquer le backstepping car le système est de forme triangulaire :
\[ \left\{ \begin{matrix}
\dot{x_1} &= x_2\\
\dot{x_2} &= -g + \frac{k}{m} \frac{x_3^2}{(c-x_1)^2}\\
\dot{x_3} &= \frac{-x_3 + k_v u}{\tau}
\end{matrix} \right. \]

On a $u^*$ qui permet d'avoir $x_3$ qui via $\alpha_3$ donne $x_3^*$ qui lui donne $x_2$ qui via $\alpha_2$ donne $x_2^*$ qui donne $x_1$ qui lui donne $x_1^*$ via $\alpha_1$.
Peut-être.

\subsection*{Etape 1:}
On pose $V_1(x_1) = \frac{1}{2}(x_1-x_1^*)^2$ avec $x_1^* = z_*$, et, $\dot{V_1} = \alpha_1 (x_1 - x_1^*)^2$. En égalisant les deux termes, on trouve après simplification que:
\[ x_2^* = \alpha_1(x_1 - x_1^*) + \dot{x_1}^* \text{avec, } \alpha_1 <0\]

\subsection*{Etape 2:}
On pose:
\begin{align*}
V_2(x_1,x_2)& = \frac{1}{2}(x_1 - x_1^*)^2 + \frac{1}{2}(x_2 - x_2^*)^2\\
\dot{V_2}(x_1,x_2) &= \alpha_1(x_1 - x_1^*)^2 + \alpha_2(x_2 - x_2^*)^2 \text{avec, } \alpha_2 < \alpha_1 <0
\intertext{On dérive, on égalise et on injecte $\dot{x_2}$ pour trouver}
x_3^2 &= \hat{x_3}^* = (\alpha_2(x_2-x_2^*) + g + \dot{x_2}^*) \frac{m}{k} (c-x_1)^2
\end{align*} 

\subsection*{Etape 3:}
On s'intéresse ici à $\hat{x_3}$ plutot que $x_3^2$.
On pose:
\begin{align*}
V_3(x_1,x_2,\hat{x_3}) &= V_2(x_1,x_2) + \frac{1}{2}(\hat{x_3} - \hat{x_3}^*)^2\\
\dot{V_3}(x_1,x_2,\hat{x_3}) &= \dot{V_2}(x_1,x_2) + \alpha_3(\hat{x_3} - \hat{x_3}^*)^2
\intertext{On dérive, on égalise et on injecte $\dot{x_3}$ pour trouver}
u = \frac{2x_3 + \alpha_3 \tau (x_3 - \hat{x_3}^*/x_3) + \tau \dot{\hat{x_3}}^*}{2k_v}
\end{align*}

Il faut éviter que $x_3 = 0$ pour avoir i non nul? 

\subsection*{Exercice IV: Commande par modes glissants}
$\alpha_0 = 1.5, \alpha(t) = \alpha_0 + \Delta \alpha avec |\Delta\alpha| \leq 0.5$
$S = \dot{\epsilon} + \beta_0 \epsilon$

$\dot{S} = \ddot{\epsilon} + \beta_0 \dot{\epsilon}$ Poursuite assymptotique 
$u = (\alpha_0 x_2^2 + \ddot{y_c} + \dot{S} + \alpha K sign(S))$
Premier terme: mode linéarisant
Terme 3 et 4:  mode glissant

$\alpha >1  K tel que |\Delta \alpha x_2^2| < K$
\end{document}

%%% Local Variables:
%%% mode: latex
%%% TeX-master: "../main"
%%% End:
